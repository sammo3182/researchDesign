\documentclass[11pt,]{article}
\usepackage[margin=1in]{geometry}
\newcommand*{\authorfont}{\fontfamily{phv}\selectfont}
\usepackage{lmodern}
\usepackage{abstract}
\renewcommand{\abstractname}{}    % clear the title
\renewcommand{\absnamepos}{empty} % originally center
\newcommand{\blankline}{\quad\pagebreak[2]}

\providecommand{\tightlist}{%
  \setlength{\itemsep}{0pt}\setlength{\parskip}{0pt}} 
\usepackage{longtable,booktabs}

\usepackage{parskip}
\usepackage{titlesec}
\titlespacing\section{0pt}{12pt plus 4pt minus 2pt}{6pt plus 2pt minus 2pt}
\titlespacing\subsection{0pt}{12pt plus 4pt minus 2pt}{6pt plus 2pt minus 2pt}

\titleformat*{\subsubsection}{\normalsize\itshape}

\usepackage{titling}
\setlength{\droptitle}{-.25cm}

%\setlength{\parindent}{0pt}
%\setlength{\parskip}{6pt plus 2pt minus 1pt}
%\setlength{\emergencystretch}{3em}  % prevent overfull lines 

\usepackage[T1]{fontenc}
\usepackage[utf8]{inputenc}

\usepackage{fancyhdr}
\pagestyle{fancy}
\usepackage{lastpage}
\renewcommand{\headrulewidth}{0.3pt}
\renewcommand{\footrulewidth}{0.0pt} 
\lhead{}
\chead{}
\rhead{\footnotesize POLI 2000: Designing Political Research -- Fall 2017}
\lfoot{}
\cfoot{\small \thepage/\pageref*{LastPage}}
\rfoot{}

\fancypagestyle{firststyle}
{
\renewcommand{\headrulewidth}{0pt}%
   \fancyhf{}
   \fancyfoot[C]{\small \thepage/\pageref*{LastPage}}
}

%\def\labelitemi{--}
%\usepackage{enumitem}
%\setitemize[0]{leftmargin=25pt}
%\setenumerate[0]{leftmargin=25pt}




\makeatletter
\@ifpackageloaded{hyperref}{}{%
\ifxetex
  \usepackage[setpagesize=false, % page size defined by xetex
              unicode=false, % unicode breaks when used with xetex
              xetex]{hyperref}
\else
  \usepackage[unicode=true]{hyperref}
\fi
}
\@ifpackageloaded{color}{
    \PassOptionsToPackage{usenames,dvipsnames}{color}
}{%
    \usepackage[usenames,dvipsnames]{color}
}
\makeatother
\hypersetup{breaklinks=true,
            bookmarks=true,
            pdfauthor={ ()},
             pdfkeywords = {},  
            pdftitle={POLI 2000: Designing Political Research},
            colorlinks=true,
            citecolor=blue,
            urlcolor=blue,
            linkcolor=magenta,
            pdfborder={0 0 0}}
\urlstyle{same}  % don't use monospace font for urls


\setcounter{secnumdepth}{0}

\usepackage{longtable}




\usepackage{setspace}

\title{POLI 2000: Designing Political Research}
\author{Yue Hu}
\date{Fall 2017}


\begin{document}  

		\maketitle
		
	
		\thispagestyle{firststyle}

%	\thispagestyle{empty}


	\noindent \begin{tabular*}{\textwidth}{ @{\extracolsep{\fill}} lr @{\extracolsep{\fill}}}


E-mail: \texttt{\href{mailto:yue-hu-1@uiowa.edu}{\nolinkurl{yue-hu-1@uiowa.edu}}} & Web: TBD\\
Office Hours: 12:30 -- 15:30 M  &  Class Hours: 15:30 -- 16:45 MW\\
Office: 313 Shaeffer Hall  & Class Room: 105 EPB\\
	&  \\
	\hline
	\end{tabular*}
	
\vspace{2mm}
	


\section{Overview}\label{overview}

Why do candidates win elections? Why do states get involved in
international crises and wars? Why do states cooperate on transnational
problems like climate change? What explains differences in how countries
treat their citizens? What explains the choices of violent non-state
actors like terrorists? There are just some of the questions that
political scientists study. The goal of this course is for you to
understand how political scientists study these questions. This course
will introduce students to political science research and the many
different ways that research is undertaken.

This class will help students understand how social scientists study
political phenomena. You will learn how to develop causal explanations
about politics and society, develop testable research hypotheses, and
design different approaches to empirically studying these theories.
Emphasis will be on an active hands-on learning environment (TILE
classroom). You will also be able to more fully understand research that
is produced in Political Science.

\subsection{Week 1 (2017-08-21\textasciitilde{}2017-08-27):
Intro}\label{week-1-2017-08-212017-08-27-intro}

\subsection{Week 2 (2017-08-28\textasciitilde{}2017-09-03): Scientific
Research}\label{week-2-2017-08-282017-09-03-scientific-research}

\subsection{Week 3 (2017-09-04\textasciitilde{}2017-09-10): Labor
Day}\label{week-3-2017-09-042017-09-10-labor-day}

\subsection{Week 4 (2017-09-11\textasciitilde{}2017-09-17):
Concept}\label{week-4-2017-09-112017-09-17-concept}

\subsection{Week 5 (2017-09-18\textasciitilde{}2017-09-24):
Theory}\label{week-5-2017-09-182017-09-24-theory}

\subsection{Week 6 (2017-09-25\textasciitilde{}2017-10-01):
Measurement}\label{week-6-2017-09-252017-10-01-measurement}

\subsection{Week 7 (2017-10-02\textasciitilde{}2017-10-08):
Validity}\label{week-7-2017-10-022017-10-08-validity}

\subsection{Week 8 (2017-10-09\textasciitilde{}2017-10-15): Design:
Experiment}\label{week-8-2017-10-092017-10-15-design-experiment}

\subsection{Week 9 (2017-10-16\textasciitilde{}2017-10-22): Case
Description}\label{week-9-2017-10-162017-10-22-case-description}

\subsection{Week 10 (2017-10-23\textasciitilde{}2017-10-29): Case
Comparison}\label{week-10-2017-10-232017-10-29-case-comparison}

\subsection{Week 11 (2017-10-30\textasciitilde{}2017-11-05): Case
Selection}\label{week-11-2017-10-302017-11-05-case-selection}

\subsection{Week 12 (2017-11-06\textasciitilde{}2017-11-12):
Large-N}\label{week-12-2017-11-062017-11-12-large-n}

\subsection{Week 13 (2017-11-13\textasciitilde{}2017-11-19):
Profession}\label{week-13-2017-11-132017-11-19-profession}

\subsection{Week 14 (2017-11-20\textasciitilde{}2017-11-26): Thanks
Giving Week}\label{week-14-2017-11-202017-11-26-thanks-giving-week}

\subsection{Week 15 (2017-11-27\textasciitilde{}2017-12-03):
Ethnicity}\label{week-15-2017-11-272017-12-03-ethnicity}

\subsection{Week 16 (2017-12-04\textasciitilde{}2017-12-10):
Presentation}\label{week-16-2017-12-042017-12-10-presentation}

\subsection{Week 17 (2017-12-11\textasciitilde{}2017-12-17): Final
Week}\label{week-17-2017-12-112017-12-17-final-week}




\end{document}

\makeatletter
\def\@maketitle{%
  \newpage
%  \null
%  \vskip 2em%
%  \begin{center}%
  \let \footnote \thanks
    {\fontsize{18}{20}\selectfont\raggedright  \setlength{\parindent}{0pt} \@title \par}%
}
%\fi
\makeatother
