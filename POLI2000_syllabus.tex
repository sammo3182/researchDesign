\documentclass[11pt,]{article}
\usepackage[margin=1in]{geometry}
\newcommand*{\authorfont}{\fontfamily{phv}\selectfont}
\usepackage{lmodern}
\usepackage{abstract}
\renewcommand{\abstractname}{}    % clear the title
\renewcommand{\absnamepos}{empty} % originally center
\newcommand{\blankline}{\quad\pagebreak[2]}

\providecommand{\tightlist}{%
  \setlength{\itemsep}{0pt}\setlength{\parskip}{0pt}} 
\usepackage{longtable,booktabs}

\usepackage{parskip}
\usepackage{titlesec}
\titlespacing\section{0pt}{12pt plus 4pt minus 2pt}{6pt plus 2pt minus 2pt}
\titlespacing\subsection{0pt}{12pt plus 4pt minus 2pt}{6pt plus 2pt minus 2pt}

\titleformat*{\subsubsection}{\normalsize\itshape}

\usepackage{titling}
\setlength{\droptitle}{-.25cm}

%\setlength{\parindent}{0pt}
%\setlength{\parskip}{6pt plus 2pt minus 1pt}
%\setlength{\emergencystretch}{3em}  % prevent overfull lines 

\usepackage[T1]{fontenc}
\usepackage[utf8]{inputenc}

\usepackage{fancyhdr}
\pagestyle{fancy}
\usepackage{lastpage}
\renewcommand{\headrulewidth}{0.3pt}
\renewcommand{\footrulewidth}{0.0pt} 
\lhead{}
\chead{}
\rhead{\footnotesize POLI 2000: Designing Political Research -- Fall 2017}
\lfoot{}
\cfoot{\small \thepage/\pageref*{LastPage}}
\rfoot{}

\fancypagestyle{firststyle}
{
\renewcommand{\headrulewidth}{0pt}%
   \fancyhf{}
   \fancyfoot[C]{\small \thepage/\pageref*{LastPage}}
}

%\def\labelitemi{--}
%\usepackage{enumitem}
%\setitemize[0]{leftmargin=25pt}
%\setenumerate[0]{leftmargin=25pt}




\makeatletter
\@ifpackageloaded{hyperref}{}{%
\ifxetex
  \usepackage[setpagesize=false, % page size defined by xetex
              unicode=false, % unicode breaks when used with xetex
              xetex]{hyperref}
\else
  \usepackage[unicode=true]{hyperref}
\fi
}
\@ifpackageloaded{color}{
    \PassOptionsToPackage{usenames,dvipsnames}{color}
}{%
    \usepackage[usenames,dvipsnames]{color}
}
\makeatother
\hypersetup{breaklinks=true,
            bookmarks=true,
            pdfauthor={ ()},
             pdfkeywords = {},  
            pdftitle={POLI 2000: Designing Political Research},
            colorlinks=true,
            citecolor=blue,
            urlcolor=blue,
            linkcolor=magenta,
            pdfborder={0 0 0}}
\urlstyle{same}  % don't use monospace font for urls


\setcounter{secnumdepth}{0}

\usepackage{longtable}




\usepackage{setspace}

\title{POLI 2000: Designing Political Research}
\author{Yue Hu}
\date{Fall 2017}


\usepackage{amsthm}
\newtheorem{theorem}{Theorem}[section]
\newtheorem{lemma}{Lemma}[section]
\theoremstyle{definition}
\newtheorem{definition}{Definition}[section]
\newtheorem{corollary}{Corollary}[section]
\newtheorem{proposition}{Proposition}[section]
\theoremstyle{definition}
\newtheorem{example}{Example}[section]
\theoremstyle{definition}
\newtheorem{exercise}{Exercise}[section]
\theoremstyle{remark}
\newtheorem*{remark}{Remark}
\newtheorem*{solution}{Solution}
\begin{document}  

		\maketitle
		
	
		\thispagestyle{firststyle}

%	\thispagestyle{empty}


	\noindent \begin{tabular*}{\textwidth}{ @{\extracolsep{\fill}} ll @{\extracolsep{\fill}}}


E-mail: \texttt{\href{mailto:yue-hu-1@uiowa.edu}{\nolinkurl{yue-hu-1@uiowa.edu}}} & Web: \href{http://uiowa.instructure.com/courses/65855}{\tt uiowa.instructure.com/courses/65855}\\
Office Hours: 12:30 -- 15:30 M \& by Appointment  &  Class Hours: 15:30 -- 16:45 M/W\\
Office: 313 Shaeffer Hall  & Class Room: 105 EPB\\
	&  \\
	\hline
	\end{tabular*}
	
\vspace{2mm}
	


\section{Overview}\label{overview}

How do candidates win elections? Why do countries get involved in
international crises and wars? What makes a country more powerful than
the others? What explains the choices of violent non-state actors like
terrorists? There are just some of the questions that political
scientists study. The goal of this course is to show you how to research
these questions as a scholar in political science. This course will
introduce students to political science research and various ways that
social scientific research is undertaken.

\subsection{Objectives}\label{objectives}

This class will introduce core elements of political science research,
such as concept, theory, hypothesis, and evidence. You will also learn
how to develop a new theory, construct testable causal inferences, and
design different approaches to test your theories empirically. Emphasis
will be on an active hands-on learning environment and full interaction
between the instructor and students.

There are three fundamental goals of this course. First, students are
expected to understand the basic principles of scientific research and
apply them to evaluate the established or on-going projects in political
science. Second, students are expected to understand common methods in
political science research and how they can offer empirical evidence to
test theoretical inferences. Third, by taking the course, students are
expected to be able to develop their own research designs following the
principles of social science research.

\section{Requirements}\label{requirements}

I will base your grade for the course on your performance in the four
areas below. You will get a score for each component, and your total
grade will be the sum of them. The grade points will be translated to
letter grades in the following way: 93-100 A, 90-92.9 A-, 87-89.9 B+,
83-86.9 B, 80-82.9 B-, 77-79.9 C+, 73-76.9 C, 70-72.9 C-, 67-69.9 D+,
63-66.9 D, 60-62.3 D-, 59.9 or less F.

\subsection{Class Performance (10\% attendance + 10\% participation +
5\%
presentation).}\label{class-performance-10-attendance-10-participation-5-presentation.}

I will take attendance at the beginning of every class. You are given
three (free) absences (i.e., no excuse is needed), but any absences
beyond that will result in ONE WHOLE GRADE REDUCTION per absence
regardless of the excuse, except for serious medical reasons or
participating in University-sponsored events. Further, if you are part
of any University-sponsored event that may cause more than three
absences, you should let me know at least 24 hours before the fourth
absence and show me the verification that you are the formal or
necessary participant of the event. Non-University sponsored events are
not excusable absences. You should use your free absences for them. The
ratio of the times of your attendance to the total number of the classes
minus three will determine how much you will get from the 10\%
attendance grade. If the ratio is equal or above one, you will get full
10\% attendance grade. Confused? Come see me sooner rather than later if
you need clarification of this policy.

Regarding participation, I am looking for you to show that you have
fully read and critically evaluated the assigned readings (all selected
articles and chapters outside the textbooks are available in the
\href{https://uiowa.instructure.com/courses/65855/modules}{Modules tab
of ICON}) and are actively engaged in our in-class discussions. Before
each week's class, you will a quick reading guideline in ICON including
some summaries and questions (also in the
\href{(https://uiowa.instructure.com/courses/65855/modules)}{Modules}).
Hope it can give you a hint about how to read the week's readings and to
which part you should pay special attention.

An important part of class performance is in-class presentation. By the
second week, we will decide a presentation schedule. In each class day
following, we will have two of you to share your thoughts on a required
reading for a given day. (In the following schedule, you will usually
see two reading materials for each week. The Monday presenters are
required to focus on the first materia, and the Wednesday presenters
should focuses on the second one. For Week 13, the reading materials
come from the same source, I've separated the readings for the
presenters and marked them in the Schedule section.) You can sign up for
the date to present in the
\href{https://uiowa.instructure.com/calendar\#view_name=scheduler\&view_start=2017-08-30\&appointment_group_id=195}{course
calendar}(ICON-Calendar-In-Class Presentation-Click the date-Reserve).
Starting from the third Wednesday (2017-09-06), every first 10 minutes
will be your showtime. The students who sign for same day are expected
work together to prepare and conduct the presentation. The presentation
is expected to include three parts: (1) a quick summary of the article
and its relations with other materials, (2) which point in the article
impresses you the most and why, and (3) at least one question you really
want to know but the article does not spell it out. I will evaluate your
performance in each---especially the latter two---aspects (see more
details in \protect\hyperlink{id}{Rubric for In-Class Presentation}).
\emph{There is no chance for makeup presentation if you miss it.}

\subsection{In-Class Quizzes (15\%)}\label{in-class-quizzes-15}

You will get eight pop quizzes throughout the semester. Each quiz
includes 1-2 questions about the readings of the given \emph{week}.
(Hint: some questions may come from the reading guideline.) The six
highest scores of them will contribute to your final grade. Quizzes will
be administered and completed at the beginning of class. \emph{You
cannot make up any missed or failed quizzes for any reason.} Feel free
to discuss the readings with your classmates before class. However, you
cannot share your answers to the reading questions with your classmates.

\subsection{Critical Response Paper
(10\%)}\label{critical-response-paper-10}

You are expected to submit one response paper over the course of the
semester. In the paper, You need to review at least two reading
materials of a given week. (You are free to write one response paper for
your presentation week. The only prohibited week for response is Week 16
(2017-12-04/2017-12-06): Professionalization.) Moreover, I would like to
see

\begin{enumerate}
\def\labelenumi{\arabic{enumi}.}
\tightlist
\item
  One and ONLY one paragraph to summarize each material (article or
  assigned chapter).
\item
  Some discussion about what you learn from these materials relating to
  the weekly topic.
\item
  Discussing how the target materials relates to others in the given
  week.
\item
  Your perspective on the argument in the materials---do you agree with
  the authors? If yes, why are their arguments convincing for you? If
  not, why?
\end{enumerate}

The paper should be 500-1000 words. It will be graded on a 10 scale and
evaluated based on the above points (see more details in
\protect\hyperlink{id}{Rubric for Critical Response Paper}). The paper
is due at the BEGINNING of the Monday class (i.e., 15:30) of the week
the materials are going to discussed (submit to ICON). Late paper will
be penalized 1/10 of the total grade for each day it is late.

\subsection{Examination (20\%)}\label{examination-20}

There is only one exam (viz. midterm) for this course. The exam is
comprised of identification and short essay questions. The exam will be
held on 2017-10-09 at the same time as the regular lecture.

\subsection{Research Proposal (25\%)}\label{research-proposal-25}

You are required to submit a research proposal by the end of the
semester. You could write it for your degree thesis or a funding
application. The proposal is expected to include at least six parts:

\begin{enumerate}
\def\labelenumi{\arabic{enumi}.}
\tightlist
\item
  A cover page (see the
  \href{https://uiowa.instructure.com/courses/65855/files/4424694?module_item_id=1468917}{template
  on ICON}).
\item
  An introduction to your research question and why it is important.
\item
  A brief literature review about what scholars have done on this topic.
  You need to cite at least three articles published in academic
  journals in political science and discuss how they relate to your
  topic. Your citation should be in the style used by the American
  Political Science Association (see the
  \href{http://www.apsanet.org/portals/54/Files/Publications/APSAStyleManual2006.pdf}{guideline}).
\item
  A section discussing your theory and hypotheses.
\item
  A section discussing the data and method you plan to use for testing
  your theory and why they are the best choice for your project.
\item
  A section discussing the operational feasibility of your research
  design.
\end{enumerate}

Each part is worth a proportion of total grade. (See the grade
distribution in \protect\hyperlink{id}{Rubric for Research Proposal}).
You will finish the project throughout the semester with my help. Here
is the general plan and some important dates:

\begin{itemize}
\tightlist
\item
  You have half of the semester prior to the Midterm to figure out what
  \textbf{question} you want to study. Once you find it, submit it to
  ICON, and I will assess whether it is a good question for the
  proposal. You must get the research question approved by 2017-10-04.
  Late submission will be penalized 1/5 of the section for each day of
  being late.
\item
  After (or even before) your research question selection, you should
  start reviewing the existing literature and construct the
  \textbf{theoretical arguments} about it. You must submit a brief (a
  couple of paragraphs) about the theory for my approval by 2017-11-13
  (submitted in ICON). Late submission will be penalized 1/5 of the
  section for each day of being late.
\item
  The \textbf{final proposal} is due by 2017-12-11 (submitted in ICON).
  Late paper will be penalized 10/100 of the total grade for each day of
  being late.
\end{itemize}

The proposal is expected to be 1500 - 2500 words (excluding the title
and reference pages) in double-spaced, 1-inch margins, and 12 font size.
The proposal will be evaluated based on each of the above parts and the
overall writing quality (see more details in
\protect\hyperlink{id}{Rubric for Research Proposal}).

\subsection{Extra Credits}\label{extra-credits}

I offer five extra credits that can be directly contributed to your
final grade (1 extra credit = 1 score of the grade). There are two ways
to get extra credits.

\begin{enumerate}
\def\labelenumi{\arabic{enumi}.}
\item
  You can earn three extra credits by using bibliography management
  software (EndNote, Jabref, Zotero, etc. See more information about
  this type of software
  \href{https://en.wikipedia.org/wiki/Comparison_of_reference_management_software}{here}.)
  and submitting relevant bibliography files with your proposal.
\item
  You can earn two credits by gaining a Certification in Human Subjects
  Protections (CITI) in ``Group 2 - Social \& Behavioral - IRB-02''. See
  more information
  \href{https://hso.research.uiowa.edu/certifications-human-subjects-protections-citi}{here}.
  To get the credits, you need to send the CITI proof to me by the due
  of the research paper.
\end{enumerate}

\section{Required Textbook}\label{required-textbook}

Earl R Babbie. \emph{The Practice of Social Research}. 14th ed. Cengage
Learning, 2016. ISBN: Print: 9781305104945, 1305104943; eText:
9781305445567, 1305445562.

\section{Schedule}\label{schedule}

\subsection{Week 1 (2017-08-21/2017-08-23): Being a Political
Scientist}\label{week-1-2017-08-212017-08-23-being-a-political-scientist}

Robert O Keohane. ``Political Science as a Vocation''.
\emph{PS: Political Science \& Politics} 42.02 (2009), pp.~359--363.

Gary King. ``Publication, Publication''.
\emph{PS: Political Science \& Politics} 39.01 (2006), pp.~119--125.

\subsection{Week 2 (2017-08-28/2017-08-30): Being A Scientist
First!}\label{week-2-2017-08-282017-08-30-being-a-scientist-first}

Babbie (2016), pp.4--28, 113--120.

Gary King. ``Replication, Replication''.
\emph{PS: Political Science \& Politics} 28.03 (1995), pp.~444--452.

\subsection{Week 3 (2017-09-04/2017-09-06): What's A Good Question
(Labor
Day)}\label{week-3-2017-09-042017-09-06-whats-a-good-question-labor-day}

Barbara Geddes. ``Big Questions, Little Answers: How the Questions You
Choose Affect the Answer You Get''. In:
\emph{Paradigms and Sand Castles: Theory Building and Research Design in Comparative Politics}.
Ann Arbor: University of Michigan Press, 2010. Chap. 2, pp.~27--88.

\subsection{Week 4 (2017-09-11/2017-09-13): Where Does A Question Come
From?}\label{week-4-2017-09-112017-09-13-where-does-a-question-come-from}

Babbie (2016), pp.89--113.

Charles C Ragin and Lisa M Amoroso.
\emph{Constructing Social Research: The Unity and Diversity of Method}.
Pine Forge Press, 2010. (Chapter 2)

\subsection{Week 5 (2017-09-18/2017-09-20): What's A
Concept?}\label{week-5-2017-09-182017-09-20-whats-a-concept}

Babbie (2016), pp.124--135.

David Collier and Steven Levitsky. ``Democracy with Adjectives:
Conceptual Innovation in Comparative Research''. \emph{World Politics}
49.03 (1997), pp.~430--451.

\subsection{Week 6 (2017-09-25/2017-09-27): Let's Measure
Politics!}\label{week-6-2017-09-252017-09-27-lets-measure-politics}

Babbie (2016), pp.135--152, 156--180.

Robert Adcock and David Collier. ``Measurement Validity: A Shared
Standard for Qualitative and Quantitative Research''.
\emph{American Political Science Review} 33 (2001), pp.~529--546.

\subsection{Week 7 (2017-10-02/2017-10-04): Measurement in
Practice}\label{week-7-2017-10-022017-10-04-measurement-in-practice}

\emph{Due for the research question approval: 2017-10-04.}

Jason Seawright and David Collier. ``Rival Strategies of Validation
Tools for Evaluating Measures of Democracy''.
\emph{Comparative Political Studies} 47.1 (2014), pp.~111--138.

Midterm Review

\subsection{Week 8 (2017-10-09/2017-10-11): What's A
Theory?}\label{week-8-2017-10-092017-10-11-whats-a-theory}

Midterm.

Babbie (2016), pp.32--58.

\subsection{Week 9 (2017-10-16/2017-10-18): Theory and Causal
Inference}\label{week-9-2017-10-162017-10-18-theory-and-causal-inference}

John Gerring. ``Causation: A Unified Framework for the Social
Sciences''. \emph{Journal of Theoretical Politics} 17.2 (2005),
pp.~163--198.

James D. Fearon. ``Counterfactuals and Hypothesis Testing In Political
Science''. \emph{World Politics} 43.2 (1991), pp.~169--195.

\subsection{Week 10 (2017-10-23/2017-10-25): Experimenting on
Politics}\label{week-10-2017-10-232017-10-25-experimenting-on-politics}

Babbie (2016), pp.225--244.

Alan S Gerber and Donald P Green. ``Field Experiments and Natural
Experiments''. In: \emph{The Oxford Handbook of Political Science}. Ed.
by Robert E. Goodin. 2011.
\url{http://www.oxfordhandbooks.com/view/10.1093/oxfordhb/9780199604456.001.0001/oxfordhb-9780199604456-e-050?mediaType=Article}
(visited on 06/15/2017).

\subsection{Week 11 (2017-10-30/2017-11-01): Principles of Case
Study}\label{week-11-2017-10-302017-11-01-principles-of-case-study}

John Gerring. ``What is a Case Study and What is it Good for?''
\emph{American Political Science Review} 98.02 (2004), pp.~341--354.

Adam Przeworski and Henry Teune.
\emph{The Logic of Comparative Social Inquiry}. New York: Joh Wiley and
Sons, 1970. 31--39, 74--87.

\subsection{Week 12 (2017-11-06/2017-11-08): Case Study in
Practice}\label{week-12-2017-11-062017-11-08-case-study-in-practice}

Barbara Geddes. ``Big Questions, Little Answers: How the Questions You
Choose Affect the Answer You Get''. In:
\emph{Paradigms and Sand Castles: Theory Building and Research Design in Comparative Politics}.
Ann Arbor: University of Michigan Press, 2010. Chap. 2, pp.~27--88.

Andrew Bennett and Colin Elman. ``Case Study Methods in the
International Relations Subfield''. \emph{Comparative Political Studies}
40.2 (2007), pp.~170--195.

\subsection{Week 13 (2017-11-13/2017-11-15): A Glance of Other Small-N
Methods}\label{week-13-2017-11-132017-11-15-a-glance-of-other-small-n-methods}

\emph{Due for the brief of the theory: 2017-11-13.}

Babbie (2016), pp.307--320, 323--348 (Mon.), 382--408 (Wed.).

\subsection{Week 14 (2017-11-20/2017-11-22): Thanksgiving
Break}\label{week-14-2017-11-202017-11-22-thanksgiving-break}

\subsection{Week 15 (2017-11-27/2017-11-29): Understanding Large-N
Analyses}\label{week-15-2017-11-272017-11-29-understanding-large-n-analyses}

Babbie (2016), pp.411-430.

Wenfang Tang, Yue Hu and Shuai Jin. ``Affirmative Inaction: Language
Education and Labor Mobility among China's Muslim Minorities''.
\emph{Chinese Sociological Review} (4 2016), pp.~346--66.

\subsection{Week 16 (2017-12-04/2017-12-06):
Professionalization}\label{week-16-2017-12-042017-12-06-professionalization}

Babbie (2016), pp.487-507.

\href{https://www.youtube.com/watch?v=bwNBXuz2eRg}{Presentation at APSA
2014: Policy Bargaining and International Conflict}

\href{https://www.youtube.com/watch?v=Z4ISkF2H4tk}{Presentation at MPSA
2017: Trump Scenes}

\subsection{Week 17 (2017-12-11/2017-12-13): Final
Week}\label{week-17-2017-12-112017-12-13-final-week}

\emph{Due for the research proposal: 2017-12-11.}

\clearpage

\hypertarget{id}{\section{Rubric for In-Class Presentation}\label{id}}

\begin{longtable}[]{@{}lll@{}}
\toprule
\begin{minipage}[b]{0.16\columnwidth}\raggedright\strut
Item\strut
\end{minipage} & \begin{minipage}[b]{0.72\columnwidth}\raggedright\strut
Criterion\strut
\end{minipage} & \begin{minipage}[b]{0.04\columnwidth}\raggedright\strut
Grade\strut
\end{minipage}\tabularnewline
\midrule
\endhead
\begin{minipage}[t]{0.16\columnwidth}\raggedright\strut
Duration\strut
\end{minipage} & \begin{minipage}[t]{0.72\columnwidth}\raggedright\strut
5-10 mins 1; \textless{} 5 min 0.\strut
\end{minipage} & \begin{minipage}[t]{0.04\columnwidth}\raggedright\strut
X\strut
\end{minipage}\tabularnewline
\begin{minipage}[t]{0.16\columnwidth}\raggedright\strut
Summary\strut
\end{minipage} & \begin{minipage}[t]{0.72\columnwidth}\raggedright\strut
Clearly described the logic and main arguments 1; covered the main
arguments 0.5; failed to capture the main arguments 0.\strut
\end{minipage} & \begin{minipage}[t]{0.04\columnwidth}\raggedright\strut
X\strut
\end{minipage}\tabularnewline
\begin{minipage}[t]{0.16\columnwidth}\raggedright\strut
Impressive point\strut
\end{minipage} & \begin{minipage}[t]{0.72\columnwidth}\raggedright\strut
Clearly described impressive point(s) and explained why 1; mentioned the
impressive point 0.5; not discuss the point at all 0.\strut
\end{minipage} & \begin{minipage}[t]{0.04\columnwidth}\raggedright\strut
X\strut
\end{minipage}\tabularnewline
\begin{minipage}[t]{0.16\columnwidth}\raggedright\strut
Critical thinking\strut
\end{minipage} & \begin{minipage}[t]{0.72\columnwidth}\raggedright\strut
Clearly described the question and why it's important 1; posted a
question 0.5; not raise any question 0.\strut
\end{minipage} & \begin{minipage}[t]{0.04\columnwidth}\raggedright\strut
X\strut
\end{minipage}\tabularnewline
\bottomrule
\end{longtable}

\hypertarget{id}{\section{Rubric for Critical Response Paper}\label{id}}

\begin{longtable}[]{@{}lll@{}}
\toprule
\begin{minipage}[b]{0.15\columnwidth}\raggedright\strut
Item\strut
\end{minipage} & \begin{minipage}[b]{0.72\columnwidth}\raggedright\strut
Criterion\strut
\end{minipage} & \begin{minipage}[b]{0.05\columnwidth}\raggedright\strut
Grade\strut
\end{minipage}\tabularnewline
\midrule
\endhead
\begin{minipage}[t]{0.15\columnwidth}\raggedright\strut
Word Count\strut
\end{minipage} & \begin{minipage}[t]{0.72\columnwidth}\raggedright\strut
\textgreater{} 500 2; 300--500 1; \textless{}300 0\strut
\end{minipage} & \begin{minipage}[t]{0.05\columnwidth}\raggedright\strut
X\strut
\end{minipage}\tabularnewline
\begin{minipage}[t]{0.15\columnwidth}\raggedright\strut
Summary\strut
\end{minipage} & \begin{minipage}[t]{0.72\columnwidth}\raggedright\strut
Clearly describe the logic and main arguments 2; cover the main
arguments 1; fail to capture the main arguments 0\strut
\end{minipage} & \begin{minipage}[t]{0.05\columnwidth}\raggedright\strut
X\strut
\end{minipage}\tabularnewline
\begin{minipage}[t]{0.15\columnwidth}\raggedright\strut
Learned point\strut
\end{minipage} & \begin{minipage}[t]{0.72\columnwidth}\raggedright\strut
Clearly explain the learned points and their importance 2; clearly
describe the points learned 1; no learned point mentioned 0\strut
\end{minipage} & \begin{minipage}[t]{0.05\columnwidth}\raggedright\strut
X\strut
\end{minipage}\tabularnewline
\begin{minipage}[t]{0.15\columnwidth}\raggedright\strut
Relation with other materials\strut
\end{minipage} & \begin{minipage}[t]{0.72\columnwidth}\raggedright\strut
Clearly explained the relation with the topic of the week and other
materials 2; mention another material 1; only talked about the assigned
reading 0.\strut
\end{minipage} & \begin{minipage}[t]{0.05\columnwidth}\raggedright\strut
X\strut
\end{minipage}\tabularnewline
\begin{minipage}[t]{0.15\columnwidth}\raggedright\strut
Critical thinking\strut
\end{minipage} & \begin{minipage}[t]{0.72\columnwidth}\raggedright\strut
Clear opinions and why 2; have a perspective of the reading 1; no
perspective at all 0\strut
\end{minipage} & \begin{minipage}[t]{0.05\columnwidth}\raggedright\strut
X\strut
\end{minipage}\tabularnewline
\bottomrule
\end{longtable}

\clearpage

\hypertarget{id}{\section{Rubric for Research Proposal}\label{id}}

\begin{longtable}[]{@{}lll@{}}
\toprule
\begin{minipage}[b]{0.12\columnwidth}\raggedright\strut
Item\strut
\end{minipage} & \begin{minipage}[b]{0.76\columnwidth}\raggedright\strut
Criterion\strut
\end{minipage} & \begin{minipage}[b]{0.03\columnwidth}\raggedright\strut
Grade\strut
\end{minipage}\tabularnewline
\midrule
\endhead
\begin{minipage}[t]{0.12\columnwidth}\raggedright\strut
Cover Page (5\%)\strut
\end{minipage} & \begin{minipage}[t]{0.76\columnwidth}\raggedright\strut
Is the table fully filled? Is every element defined?\strut
\end{minipage} & \begin{minipage}[t]{0.03\columnwidth}\raggedright\strut
X\strut
\end{minipage}\tabularnewline
\begin{minipage}[t]{0.12\columnwidth}\raggedright\strut
Research Question Approval (5\%)\strut
\end{minipage} & \begin{minipage}[t]{0.76\columnwidth}\raggedright\strut
Was the research proposal approved by 2017-10-09?\strut
\end{minipage} & \begin{minipage}[t]{0.03\columnwidth}\raggedright\strut
X\strut
\end{minipage}\tabularnewline
\begin{minipage}[t]{0.12\columnwidth}\raggedright\strut
Brief of Theory (5\%)\strut
\end{minipage} & \begin{minipage}[t]{0.76\columnwidth}\raggedright\strut
Was the brief of the theory submitted by 2017-11-12?\strut
\end{minipage} & \begin{minipage}[t]{0.03\columnwidth}\raggedright\strut
X\strut
\end{minipage}\tabularnewline
\begin{minipage}[t]{0.12\columnwidth}\raggedright\strut
Introduction (10\%)\strut
\end{minipage} & \begin{minipage}[t]{0.76\columnwidth}\raggedright\strut
Is the research question well stated? Does the intro clearly explain the
importance of the study? Does the intro clearly explain the potential
contribution of this project?\strut
\end{minipage} & \begin{minipage}[t]{0.03\columnwidth}\raggedright\strut
X\strut
\end{minipage}\tabularnewline
\begin{minipage}[t]{0.12\columnwidth}\raggedright\strut
Literature Review (10\%)\strut
\end{minipage} & \begin{minipage}[t]{0.76\columnwidth}\raggedright\strut
Does the LR address more than three existing studies? Does the LR
clearly review the findings of the existing literature? Does the LR
clearly state how the existing literature serve as the basis for this
study?\strut
\end{minipage} & \begin{minipage}[t]{0.03\columnwidth}\raggedright\strut
X\strut
\end{minipage}\tabularnewline
\begin{minipage}[t]{0.12\columnwidth}\raggedright\strut
Theory (15\%)\strut
\end{minipage} & \begin{minipage}[t]{0.76\columnwidth}\raggedright\strut
Is the causal logic clearly stated? Are the concepts in the theory
defined well? Is the causal chain complete and consistent? Are the
causal inferences (hypotheses) clearly stated and consistent with the
theory?\strut
\end{minipage} & \begin{minipage}[t]{0.03\columnwidth}\raggedright\strut
X\strut
\end{minipage}\tabularnewline
\begin{minipage}[t]{0.12\columnwidth}\raggedright\strut
Research Design (20\%)\strut
\end{minipage} & \begin{minipage}[t]{0.76\columnwidth}\raggedright\strut
Does the author clearly describe the strategy to test the hypotheses?
Does the author defend his/her method choice well?\strut
\end{minipage} & \begin{minipage}[t]{0.03\columnwidth}\raggedright\strut
X\strut
\end{minipage}\tabularnewline
\begin{minipage}[t]{0.12\columnwidth}\raggedright\strut
Data (10\%)\strut
\end{minipage} & \begin{minipage}[t]{0.76\columnwidth}\raggedright\strut
Is there a complete plan of data collection? How do the data fit the
research design? Validations?\strut
\end{minipage} & \begin{minipage}[t]{0.03\columnwidth}\raggedright\strut
X\strut
\end{minipage}\tabularnewline
\begin{minipage}[t]{0.12\columnwidth}\raggedright\strut
Feasibility (5\%)\strut
\end{minipage} & \begin{minipage}[t]{0.76\columnwidth}\raggedright\strut
Is the research design a feasible one for a college student? What are
the potential difficulties the researcher may encounter?\strut
\end{minipage} & \begin{minipage}[t]{0.03\columnwidth}\raggedright\strut
X\strut
\end{minipage}\tabularnewline
\begin{minipage}[t]{0.12\columnwidth}\raggedright\strut
Citation (10\%)\strut
\end{minipage} & \begin{minipage}[t]{0.76\columnwidth}\raggedright\strut
Are the citations well presented? Is there a full bibliography attached?
Are the citation and bibliography styles consistent with the APSR
requirement?\strut
\end{minipage} & \begin{minipage}[t]{0.03\columnwidth}\raggedright\strut
X\strut
\end{minipage}\tabularnewline
\begin{minipage}[t]{0.12\columnwidth}\raggedright\strut
Overall writing (10\%)\strut
\end{minipage} & \begin{minipage}[t]{0.76\columnwidth}\raggedright\strut
Was the proposal proofread and edited? Are the paragraphs well framed
and organized? Do the word count and layout match the requirement?\strut
\end{minipage} & \begin{minipage}[t]{0.03\columnwidth}\raggedright\strut
X\strut
\end{minipage}\tabularnewline
\bottomrule
\end{longtable}

\clearpage

\section{CLAS Teaching Policies \& Resources --- Syllabus
Insert}\label{clas-teaching-policies-resources-syllabus-insert}

\subsection{Administrative Home}\label{administrative-home}

The College of Liberal Arts and Sciences is the administrative home of
this course and governs matters such as the add/drop deadlines, the
second-grade-only option, and other related issues. Different colleges
may have different policies. Questions may be addressed to 120 Schaeffer
Hall, or see the CLAS Academic Policies Handbook at
\url{https://clas.uiowa.edu/students/handbook}.

\subsection{Electronic Communication}\label{electronic-communication}

University policy specifies that students are responsible for all
official correspondences sent to their University of Iowa e-mail address
(@ uiowa.edu). Faculty and students should use this account for
correspondences
(\href{https://opsmanual.uiowa.edu/human-resources/professional-ethics-and-academic-responsibility\#15.2}{Operations
Manual, III.15.2}, k.11).

\subsection{Accommodations for
Disabilities}\label{accommodations-for-disabilities}

The University of Iowa is committed to providing an educational
experience that is accessible to all students. A student may request
academic accommodations for a disability (which includes but is not
limited to mental health, attention, learning, vision, and physical or
health-related conditions). A student seeking academic accommodations
should first register with Student Disability Services and then meet
with the course instructor privately in the instructor's office to make
particular arrangements. Reasonable accommodations are established
through an interactive process between the student, instructor, and SDS.
See \url{https://sds.studentlife.uiowa.edu/} for information.

\subsection{Nondiscrimination in the
Classroom}\label{nondiscrimination-in-the-classroom}

The University of Iowa is committed to making the classroom a respectful
and inclusive space for all people irrespective of their gender, sexual,
racial, religious or other identities. Toward this goal, students are
invited to optionally share their preferred names and pronouns with
their instructors and classmates. The University of Iowa prohibits
discrimination and harassment against individuals on the basis of race,
class, gender, sexual orientation, national origin, and other identity
categories set forth in the University's Human Rights policy. For more
information, contact the Office of Equal Opportunity and Diversity,
\href{mailto:diversity@uiowa.edu}{\nolinkurl{diversity@uiowa.edu}}, or
visit
\href{https://diversity.uiowa.edu/office/equal-opportunity-and-diversity}{diversity.uiowa.edu}.

\subsection{Academic Honesty}\label{academic-honesty}

All CLAS students or students taking classes offered by CLAS have, in
essence, agreed to the
\href{https://clas.uiowa.edu/students/handbook/academic-fraud-honor-code}{College's
Code of Academic Honesty}: ``I pledge to do my own academic work and to
excel to the best of my abilities, upholding the
\href{https://newstudents.uiowa.edu/iowa-challenge}{IOWA Challenge}. I
promise not to lie about my academic work, to cheat, or to steal the
words or ideas of others; nor will I help fellow students to violate the
Code of Academic Honesty.'' Any student committing academic misconduct
is reported to the College and placed on disciplinary probation or may
be suspended or expelled
(\href{https://clas.uiowa.edu/students/handbook}{CLAS Academic Policies
Handbook}).

\subsection{CLAS Final Examination
Policies}\label{clas-final-examination-policies}

The final examination schedule for each class is announced by the
Registrar generally by the fifth week of classes. Final exams are
offered only during the official final examination period. No exams of
any kind are allowed during the last week of classes. All students
should plan on being at the UI through the final examination period.
Once the Registrar has announced the date, time, and location of each
final exam, the complete schedule will be published on the Registrar's
web site and will be shared with instructors and students. It is the
student's responsibility to know the date, time, and place of a final
exam.

\subsection{Making a Suggestion or a
Complaint}\label{making-a-suggestion-or-a-complaint}

Students with a suggestion or complaint should first visit with the
instructor (and the course supervisor), and then with the departmental
DEO (Wenfang Tang, 335-2358). Complaints must be made within six months
of the incident (\href{https://clas.uiowa.edu/students/handbook}{CLAS
Academic Policies Handbook}).

\subsection{Understanding Sexual
Harassment}\label{understanding-sexual-harassment}

Sexual harassment subverts the mission of the University and threatens
the well-being of students, faculty, and staff. All members of the UI
community have a responsibility to uphold this mission and to contribute
to a safe environment that enhances learning. Incidents of sexual
harassment should be reported immediately. See the UI
\href{https://osmrc.uiowa.edu/}{Office of the Sexual Misconduct Response
Coordinator} for assistance, definitions, and the full University
policy.

\subsection{Reacting Safely to Severe
Weather}\label{reacting-safely-to-severe-weather}

In severe weather, class members should seek appropriate shelter
immediately, leaving the classroom if necessary. The class will continue
if possible when the event is over. For more information on Hawk Alert
and the siren warning system, visit the
\href{https://police.uiowa.edu/emergency-communications}{Department of
Public Safety website}.

\clearpage

\section*{Reference}\label{reference}
\addcontentsline{toc}{section}{Reference}

\hypertarget{refs}{}
\hypertarget{ref-Babbie2016}{}
Babbie, Earl R. 2016. \emph{The Practice of Social Research}. 14th ed.
Cengage Learning.



\clearpage
\end{document}

\makeatletter
\def\@maketitle{%
  \newpage
%  \null
%  \vskip 2em%
%  \begin{center}%
  \let \footnote \thanks
    {\fontsize{18}{20}\selectfont\raggedright  \setlength{\parindent}{0pt} \@title \par}%
}
%\fi
\makeatother
