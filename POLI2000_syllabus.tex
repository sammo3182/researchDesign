\documentclass[11pt,]{article}
\usepackage[margin=1in]{geometry}
\newcommand*{\authorfont}{\fontfamily{phv}\selectfont}
\usepackage{lmodern}
\usepackage{abstract}
\renewcommand{\abstractname}{}    % clear the title
\renewcommand{\absnamepos}{empty} % originally center
\newcommand{\blankline}{\quad\pagebreak[2]}

\providecommand{\tightlist}{%
  \setlength{\itemsep}{0pt}\setlength{\parskip}{0pt}} 
\usepackage{longtable,booktabs}

\usepackage{parskip}
\usepackage{titlesec}
\titlespacing\section{0pt}{12pt plus 4pt minus 2pt}{6pt plus 2pt minus 2pt}
\titlespacing\subsection{0pt}{12pt plus 4pt minus 2pt}{6pt plus 2pt minus 2pt}

\titleformat*{\subsubsection}{\normalsize\itshape}

\usepackage{titling}
\setlength{\droptitle}{-.25cm}

%\setlength{\parindent}{0pt}
%\setlength{\parskip}{6pt plus 2pt minus 1pt}
%\setlength{\emergencystretch}{3em}  % prevent overfull lines 

\usepackage[T1]{fontenc}
\usepackage[utf8]{inputenc}

\usepackage{fancyhdr}
\pagestyle{fancy}
\usepackage{lastpage}
\renewcommand{\headrulewidth}{0.3pt}
\renewcommand{\footrulewidth}{0.0pt} 
\lhead{}
\chead{}
\rhead{\footnotesize POLI 2000: Designing Political Research -- Fall 2017}
\lfoot{}
\cfoot{\small \thepage/\pageref*{LastPage}}
\rfoot{}

\fancypagestyle{firststyle}
{
\renewcommand{\headrulewidth}{0pt}%
   \fancyhf{}
   \fancyfoot[C]{\small \thepage/\pageref*{LastPage}}
}

%\def\labelitemi{--}
%\usepackage{enumitem}
%\setitemize[0]{leftmargin=25pt}
%\setenumerate[0]{leftmargin=25pt}




\makeatletter
\@ifpackageloaded{hyperref}{}{%
\ifxetex
  \usepackage[setpagesize=false, % page size defined by xetex
              unicode=false, % unicode breaks when used with xetex
              xetex]{hyperref}
\else
  \usepackage[unicode=true]{hyperref}
\fi
}
\@ifpackageloaded{color}{
    \PassOptionsToPackage{usenames,dvipsnames}{color}
}{%
    \usepackage[usenames,dvipsnames]{color}
}
\makeatother
\hypersetup{breaklinks=true,
            bookmarks=true,
            pdfauthor={ ()},
             pdfkeywords = {},  
            pdftitle={POLI 2000: Designing Political Research},
            colorlinks=true,
            citecolor=blue,
            urlcolor=blue,
            linkcolor=magenta,
            pdfborder={0 0 0}}
\urlstyle{same}  % don't use monospace font for urls


\setcounter{secnumdepth}{0}

\usepackage{longtable}




\usepackage{setspace}

\title{POLI 2000: Designing Political Research}
\author{Yue Hu}
\date{Fall 2017}


\usepackage{amsthm}
\newtheorem{theorem}{Theorem}[section]
\newtheorem{lemma}{Lemma}[section]
\theoremstyle{definition}
\newtheorem{definition}{Definition}[section]
\newtheorem{corollary}{Corollary}[section]
\newtheorem{proposition}{Proposition}[section]
\theoremstyle{definition}
\newtheorem{example}{Example}[section]
\theoremstyle{remark}
\newtheorem*{remark}{Remark}
\begin{document}  

		\maketitle
		
	
		\thispagestyle{firststyle}

%	\thispagestyle{empty}


	\noindent \begin{tabular*}{\textwidth}{ @{\extracolsep{\fill}} lr @{\extracolsep{\fill}}}


E-mail: \texttt{\href{mailto:yue-hu-1@uiowa.edu}{\nolinkurl{yue-hu-1@uiowa.edu}}} & Web: TBD\\
Office Hours: 12:30 -- 15:30 M \& by Appointment  &  Class Hours: 15:30 -- 16:45 M/W\\
Office: 313 Shaeffer Hall  & Class Room: 105 EPB\\
	&  \\
	\hline
	\end{tabular*}
	
\vspace{2mm}
	


\section{Overview}\label{overview}

How do candidates win elections? Why do countries get involved in
international crises and wars? What makes a country more powerful than
the others? What explains the choices of violent non-state actors like
terrorists? There are just some of the questions that political
scientists study. The goal of this course is show you how to research
these questions as a scholar in political science. This course will
introduce students to political science research and various ways that
social scientific research is undertaken.

This class will introduce the core elements of a political science
research, such as concept, theory, hypothesis, and evidence. You will
also learn how to build a new theory, develop testable causal
inferences, and design different approaches to test your theories
empirically. Emphasis will be on an active hands-on learning environment
and fully interactions between the instructor and students. Students can
expect to comprehend the research produced in Political Science and
learn to conduct their own..

\section{Requirements}\label{requirements}

I will base your grade for the course on your performance in the four
areas below. You will get a score for each component, and your total
grade will be the sum of them. The grade points will be translated to
letter grades in the following way: 93-100 A, 90-92.9 A-, 87-89.9 B+,
83-86.9 B, 80-82.9 B-, 77-79.9 C+, 73-76.9 C, 70-72.9 C-, 67-69.9 D+,
63-66.9 D, 60-62.3 D-, 59.9 or less F.

\subsection{Class Performance (10\% attendance + 10\% participation +
5\%
presentation).}\label{class-performance-10-attendance-10-participation-5-presentation.}

Regarding participation, I am looking for you to show that you have
fully read and critically evaluated the assigned readings (all available
in ICON) and are actively engaged with our in-class discussions. Before
each week's class you will receive a one-page reading guideline. It will
give you a hint how to read the week's readings and which part you
should pay special attention to. The guideline will always include a
recommanded reading order and several questions to help you comprehend
the materials.

By the second week, we will decide a presentatoin schedule together. You
will sign in a schedule sheet by picking the topic and articles you are
interested. Starting from the third Monday, every first 20 minutes will
be your show time. The student who signed for that day will make a
10-min presentation about the reading. The presentation should include
three parts: a brief summary about what the article is about and its
relations with other materials, which point in the article impresses you
the most and why, and at least one question you really want to know but
the article does not spell it out. I will evaluate your performance in
each---especially the latter two---aspects (see more details in the
attached rubric). Please cherish this chance to practice your
presentation capability, and there is no chance for make-up presentation
if you miss it.

\subsection{In-class quizzes (10\%)}\label{in-class-quizzes-10}

You will get ten pop quizzes throughout the semester. Each quiz includes
2-3 questions about the required readings we are going to discuss in the
class. (Hint: some questions may come from the reading guide.) At the
end of the semester, the eight highest scores of the ten quizzes will be
recorded to contribute to your final grade. Quizzes will be administered
and completed either at the beginning or the end of class. You cannot
make up any missed or failed quizzes for any reason. Feel free to
discuss the readings with your classmates prior to class. \emph{However,
you cannot share your answers to the reading questions with your
classmates.}

\subsection{Critical Response paper
(12\%)}\label{critical-response-paper-12}

You are expected to submit two response papers for weekly readings. You
are free to pick any week's readings you like as long as we haven't
discussed them yet. (You are free to write one response paper for your
presentation week.) In the paper, You need to review at least three
reading materials of the week. Moreover, I would like to see

\begin{enumerate}
\def\labelenumi{\arabic{enumi}.}
\tightlist
\item
  One and ONLY one paragraphs to summarize each material.
\item
  Some discussion about what do you learn from these materials relating
  to the week topic.
\item
  Your perspective about the argument in the materials---do you agree
  with the author? If yes, why are his/her arguments convincing for you?
  If not, why?
\item
  Cite other sources if appropriate (but not required).
\end{enumerate}

The paper is due by the BEGINNING of the Monday class (i.e., 15:30) of
the week the materils are going to discussed (submit to ICON). Late
paper will be penalized 10 points for each day of being late. Each paper
should be 2-6 pages, double-spaced, 1 inch margins, and in 12 font size.
Papers will be evaluate based on the above points (see more details in
the attached rubric).

\subsection{Examination (20\%)}\label{examination-20}

There is only one exam (viz. midterm) for this course. The exam is
comprised of identification and short essay questions. The exam will be
held on 2017-10-09 in the same class of the lecture.

\subsection{Research Proposal (23\%)}\label{research-proposal-23}

You are expected to submit a research proposal at the end of the
semester. It could be a proposal for your degree thesis or for a funding
application. The proposal should include at least six parts:

\begin{enumerate}
\def\labelenumi{\arabic{enumi}.}
\tightlist
\item
  A cover table (see template on ICON).
\item
  An introduction of your research question and why it is important.
\item
  A brief literature review about what scholars have done on this topic.
  You need to cite at least three articles published in academic
  journals of political science in this part and discuss how they
  relates to your topic. Your citation should be in the style used by
  the American Political Science Association (see the guideline
  \href{http://www.apsanet.org/portals/54/Files/Publications/APSAStyleManual2006.pdf}{here}).
\item
  A section discussing your theory and hypotheses.
\item
  A section discussing the data and method you plan to use to test your
  theory and why they are the best choice for your project.
\item
  A section discussing the operational feasibility of your research
  design.
\end{enumerate}

The proposal is expected 5-10 pages (excluding the title and reference
pages) in double-spaced, 1 inch margins, and 12 font size. The proposal
will be evaluated based on each of the above parts and the overall
writing (see more details in the attached rubric).

Here are some important dates relating to the research proposal:

\begin{itemize}
\tightlist
\item
  You must get your research question approved by the instructor by
  2017-10-09 (submitted in ICON).
\item
  You must submit a brief about your theory by 2017-11-12 (submitted in
  ICON).
\item
  The proposal is due by 2017-12-11. Late paper will be penalized 30
  points for each day of being late (submitted in ICON).
\end{itemize}

\subsubsection{Extra credit}\label{extra-credit}

You can get 2 extra credits if you use bibliography management softwares
(Endnote, Jabref, Zotero, etc. see more information about these
softwares
\href{https://en.wikipedia.org/wiki/Comparison_of_reference_management_software}{here}.)
and submit relevant bibliography files with your proposal.

You can get another 2 credits if you can gain the Certifications in
Human Subjects Protections (CITI) in ``Group 2 - Social \& Behavioral -
IRB-02''. See more information about it
\href{https://hso.research.uiowa.edu/certifications-human-subjects-protections-citi}{here}.

You can get a third extra credits by attending an academic workshop held
by the Department of Political Science and send me a selfie at the
workshop scence.

\section{Required textbook:}\label{required-textbook}

Earl R. Babbie. \emph{The Practice of Social Research}. 13th ed.
Australia: Wadsworth Cengage Learning, 2012. ISBN: 9781133049791
1133049796.

S. Van Evera. \emph{Guide to Methods for Students of Political Science}.
Cornell paperbacks. Cornell University Press, 1997. ISBN: 9780801484575.

\section{Schedule}\label{schedule}

\subsection{Week 1 (2017-08-21/2017-08-23): Being a Political
Scientist}\label{week-1-2017-08-212017-08-23-being-a-political-scientist}

Robert O Keohane. ``Political Science as a Vocation''.
\emph{PS: Political Science and Politics} 42.02 (2009), pp.~359--363.

John S Dryzek. ``Revolutions without Enemies: Key Transformations in
Political Science''. \emph{American Political Science Review} 100.04
(2006), pp.~487--92.

Gary King. ``Publication, Publication''.
\emph{PS: Political Science and Politics} 39.01 (2006), pp.~119--125.

\subsection{Week 2 (2017-08-28/2017-08-30): Being
Scientific}\label{week-2-2017-08-282017-08-30-being-scientific}

Babbie (2012), pp.1-27, 112-120.

Gary King. ``Replication, Replication''.
\emph{PS: Political Science and Politics} 28.03 (1995), pp.~444--452.

Gabriel A Almond. ``Separate Tables: Schools and Sects in Political
Science''. \emph{PS: Political Science and Politics} 21.4 (1988),
pp.~828--42.

Imre Lakatos and Musgrave Alan. ``Falsification and the Methodology of
Scientific Research Programmes''.
\emph{Criticism and the Growth of Knowledge} (1970), pp.~91--180.

\subsection{Week 3 (2017-09-04/2017-09-06): What's A Good Question
(Labor
Day)}\label{week-3-2017-09-042017-09-06-whats-a-good-question-labor-day}

Barbara Geddes. ``Big Questions, Little Answers: How the Questions You
Choose Affect the Answer You Get''. In:
\emph{Paradigms and Sand Castles: Theory Building and Research Design in Comparative Politics}.
Ann Arbor: University of Michigan Press, 2010. Chap. 2, pp.~27--88.

Van Evera (1997), pp.97-99.

\subsection{Week 4 (2017-09-11/2017-09-13): How to Find Research
Question}\label{week-4-2017-09-112017-09-13-how-to-find-research-question}

Babbie (2012), pp.91-112.

Efren O Perez and Margit Tavits. ``Language Shapes People's Time
Perspective and Support for Future-Oriented Policies''.
\emph{American Journal of Political Science} (2017), pp.~1--13.

Timothy J McKeown. ``Case Studies and the Statistical Worldview Review
of King, Keohane, and Verba's Designing Social Inquiry Scientific
Inference in Qualitative Research''. \emph{International organization}
53.01 (1999), pp.~161--190.

Charles C Ragin and Lisa M Amoroso.
\emph{Constructing Social Research: The Unity and Diversity of Method}.
Pine Forge Press, 2010. (Chapter 1, 2)

\subsection{Week 5 (2017-09-18/2017-09-20):
Concepts}\label{week-5-2017-09-182017-09-20-concepts}

Babbie (2012), pp.165-177.

Michael Barnett and Raymond Duvall. ``Power in International Politics''.
\emph{International Organization} 59.01 (2005), pp.~39--75.

David Collier and Steven Levitsky. ``Democracy with Adjectives:
Conceptual Innovation in Comparative Research''. \emph{World Politics}
49.03 (1997), pp.~430--451.

David Collier and James E Mahon. ``Conceptual ``Stretching'' Revisited:
Adapting Categories in Comparative Analysis''.
\emph{American Political Science Review} 87.04 (1993), pp.~845--855.

Giovanni Sartori. ``Concept Misformation in Comparative Politics''.
\emph{American Political Science Review} 64.04 (1970), pp.~1033--1053.

\subsection{Week 6 (2017-09-25/2017-09-27): Principles of
Measurement}\label{week-6-2017-09-252017-09-27-principles-of-measurement}

Babbie (2012), pp.177-194, 197-223.

Jason Seawright and David Collier. ``Rival Strategies of Validation
Tools for Evaluating Measures of Democracy''.
\emph{Comparative Political Studies} 47.1 (2014), pp.~111--138.

Andreas Schedler. ``Judgment and Measurement in Political Science''.
\emph{Perspectives on Politics} 10.1 (2012), pp.~21--36.

Shawn Treier and Simon Jackman. ``Democracy as a Latent Variable''.
\emph{American Journal of Political Science} 52.1 (2008), pp.~201--217.

Robert Adcock and David Collier. ``Measurement Validity: A Shared
Standard for Qualitative and Quantitative Research''.
\emph{American Political Science Review} 33 (2001), pp.~529--546.

\subsection{Week 7 (2017-10-02/2017-10-04): Measurement in
Practice}\label{week-7-2017-10-022017-10-04-measurement-in-practice}

Van Evera (1997), pp.7-50.

Midterm Review

\subsection{Week 8 (2017-10-09/2017-10-11): What's
Theory?}\label{week-8-2017-10-092017-10-11-whats-theory}

Due for the research

Midterm.

Babbie (2012), pp.57-83.

John Gerring. ``Causation: A Unified Framework for the Social
Sciences''. \emph{Journal of Theoretical Politics} 17.2 (2005),
pp.~163--198.

Adam Przeworski and Henry Teune.
\emph{The Logic of Comparative Social Inquiry}. New York: Joh Wiley and
Sons, 1970. Chapter 1.

\subsection{Week 9 (2017-10-16/2017-10-18): Theory and Causal
Inference}\label{week-9-2017-10-162017-10-18-theory-and-causal-inference}

James Johnson. ``How Conceptual Problems Migrate: Rational Choice,
Interpretation, and the Hazards of Pluralism''.
\emph{Annual Review of Political Science} 5.1 (Jun. 17, 2017),
pp.~223--48. (Visited on 06/17/2017).

John Gerring. ``Causal mechanisms: Yes, But\ldots{}''
\emph{Comparative Political Studies} 43.11 (2010), pp.~1499--526.

Steven Bernstein, Richard Ned Lebow, Janice Gross Stein and Steven
Weber. ``God Gave Physics the Easy Problems: Adapting Social Science to
an Unpredictable World''.
\emph{European Journal of International Relations} 6.1 (2000),
pp.~43--76.

James D. Fearon. ``Counterfactuals and Hypothesis Testing In Political
Science''. \emph{World Politics} 43.2 (1991), pp.~169--195.

\subsection{Week 10 (2017-10-23/2017-10-25): Experimental Logic and
Design}\label{week-10-2017-10-232017-10-25-experimental-logic-and-design}

Babbie (2012), pp.271-291.

Alan S Gerber and Donald P Green. ``Field Experiments and Natural
Experiments''. In: \emph{The Oxford Handbook of Political Science}. Ed.
by Robert E. Goodin. 2011.
\url{http://www.oxfordhandbooks.com/view/10.1093/oxfordhb/9780199604456.001.0001/oxfordhb-9780199604456-e-050?mediaType=Article}
(visited on 06/15/2017).

Alex Mintz, Steven B Redd and Arnold Vedlitz. ``Can We Generalize from
Student Experiments to the Real World in Political Science, Military
Affairs, and International Relations?''
\emph{Journal of Conflict Resolution} 50.5 (2006), pp.~757--776.

Alex Mintz. ``Foreign Policy Decision Making in Familiar and Unfamiliar
Settings''. \emph{Journal of Conflict Resolution} 48.1 (2004),
pp.~91--104.

\subsection{Week 11 (2017-10-30/2017-11-01): Principles of Case
Study}\label{week-11-2017-10-302017-11-01-principles-of-case-study}

Jack S Levy. ``Case Studies: Types, Designs, and Logics of Inference''.
\emph{Conflict Management and Peace Science} 25.1 (2008), pp.~1--18.

John Gerring. ``What is a Case Study and What is it Good for?''
\emph{American Political Science Review} 98.02 (2004), pp.~341--354.

Timothy J McKeown. ``Case Studies and the Limits of the Quantitative
Worldview''. In:
\emph{Rethinking Social Inquiry: Diverse Tools, Shared Standards}. Ed.
by David Collier and Henry E. Brady. Lanham, MD: Rowman and Littlefield,
2004, pp.~139--167.

Adam Przeworski and Henry Teune.
\emph{The Logic of Comparative Social Inquiry}. New York: Joh Wiley and
Sons, 1970. 31-39, 74-87.

\subsection{Week 12 (2017-11-06/2017-11-08): Case Study in
Practice}\label{week-12-2017-11-062017-11-08-case-study-in-practice}

Van Evera (1997), pp.49-88.

Barbara Geddes. ``How the Cases You Choose Affect the Answers You Get:
Selection Bias in Comparative Politics''. In:
\emph{Paradigms and Sand Castles: Theory Building and Research Design in Comparative Politics}.
Ann Arbor: University of Michigan Press, 2010, pp.~89--129.

Andrew Bennett and Colin Elman. ``Case Study Methods in the
International Relations Subfield''. \emph{Comparative Political Studies}
40.2 (2007), pp.~170--195.

Gerardo L. Munck. ``Tools for Qualitative Research''. In:
\emph{Rethinking Social Inquiry: Diverse Tools, Shared Standards}. Ed.
by David Collier and Henry E. Brady. Lanham, MD: Rowman and Littlefield,
2004, pp.~105--121.

\subsection{Week 13 (2017-11-13/2017-11-15): A Glance of Other Small-N
Methods}\label{week-13-2017-11-132017-11-15-a-glance-of-other-small-n-methods}

Babbie (2012), pp.295-321.

Andrew Bennett. ``Process Tracing: A Bayesian Perspective''. In:
\emph{Oxford Handbook of Political Methodology}. Ed. by Janet Box
Steffensmeier, Henry Brady and David Coiller. 2008, pp.~702--21.

Giovanni Capoccia and R Daniel Kelemen. ``The Study of Critical
Junctures: Theory, Narrative, and Counterfactuals in Historical
Institutionalism''. \emph{World Politics} 59.03 (2007), pp.~341--369.

Clifford Geertz. ``Thick Description: Toward an Interpretive Theory of
Culture''. \emph{Readings in the Philosophy of Social Science} (1994),
pp.~213--31.

\subsection{Week 14 (2017-11-20/2017-11-22): Thanks Giving
Week}\label{week-14-2017-11-202017-11-22-thanks-giving-week}

\subsection{Week 15 (2017-11-27/2017-11-29): Understanding Large-N
Analyses}\label{week-15-2017-11-272017-11-29-understanding-large-n-analyses}

Babbie (2012), pp.415-438.

Wenfang Tang, Yue Hu and Shuai Jin. ``Affirmative Inaction: Language
Education and Labor Mobility among China's Muslim Minorities''.
\emph{Chinese Sociological Review} (4 2016), pp.~346--66.

Emilie M Hafner-Burton and Alexander H Montgomery. ``Power Positions:
International Organizations, Social Networks, and Conflict''.
\emph{Journal of Conflict Resolution} 50.1 (2006), pp.~3--27.

James Mahoney and Gary Goertz. ``A Tale of Two Cultures: Contrasting
Quantitative and Qualitative Research''. \emph{Political Analysis} 14.3
(2006), pp.~227--49.

\subsection{Week 16 (2017-12-04/2017-12-06):
Professionalization}\label{week-16-2017-12-042017-12-06-professionalization}

Babbie (2012), pp.498-519.

Van Evera (1997), pp.99-111.

\href{https://www.youtube.com/watch?v=bwNBXuz2eRg}{Presentation at APSA
2014: Policy Bargaining and International Conflict}

\href{https://www.youtube.com/watch?v=Z4ISkF2H4tk}{Presentation at MPSA
2017: Trump Scenes}

\subsection{Week 17 (2017-12-11/2017-12-13): Final
Week}\label{week-17-2017-12-112017-12-13-final-week}

\clearpage

\section{Rubric for In-Class
Presentation}\label{rubric-for-in-class-presentation}

\begin{longtable}[]{@{}lll@{}}
\toprule
\begin{minipage}[b]{0.16\columnwidth}\raggedright\strut
Item\strut
\end{minipage} & \begin{minipage}[b]{0.72\columnwidth}\raggedright\strut
Criteron\strut
\end{minipage} & \begin{minipage}[b]{0.04\columnwidth}\raggedright\strut
Grade\strut
\end{minipage}\tabularnewline
\midrule
\endhead
\begin{minipage}[t]{0.16\columnwidth}\raggedright\strut
Duration\strut
\end{minipage} & \begin{minipage}[t]{0.72\columnwidth}\raggedright\strut
\textgreater{} 8 mins 1; 5-8 mins .5; \textless{} 5 min 0.\strut
\end{minipage} & \begin{minipage}[t]{0.04\columnwidth}\raggedright\strut
X\strut
\end{minipage}\tabularnewline
\begin{minipage}[t]{0.16\columnwidth}\raggedright\strut
Summary of the material\strut
\end{minipage} & \begin{minipage}[t]{0.72\columnwidth}\raggedright\strut
clearly described the logic and main arguments 1; covered the main
arguments 0.5; failed to capture the main arguments 0.\strut
\end{minipage} & \begin{minipage}[t]{0.04\columnwidth}\raggedright\strut
X\strut
\end{minipage}\tabularnewline
\begin{minipage}[t]{0.16\columnwidth}\raggedright\strut
Relation with other materials\strut
\end{minipage} & \begin{minipage}[t]{0.72\columnwidth}\raggedright\strut
clearly explained the relation with the week topic and other materials
1; mention another material 0.5; only talked about the assigned reading
0.\strut
\end{minipage} & \begin{minipage}[t]{0.04\columnwidth}\raggedright\strut
X\strut
\end{minipage}\tabularnewline
\begin{minipage}[t]{0.16\columnwidth}\raggedright\strut
Impressive point\strut
\end{minipage} & \begin{minipage}[t]{0.72\columnwidth}\raggedright\strut
clearly described the impressive point and explained why 1; mentioned
the impressive point 0.5; not discuss the point at all 0.\strut
\end{minipage} & \begin{minipage}[t]{0.04\columnwidth}\raggedright\strut
X\strut
\end{minipage}\tabularnewline
\begin{minipage}[t]{0.16\columnwidth}\raggedright\strut
Critical reading\strut
\end{minipage} & \begin{minipage}[t]{0.72\columnwidth}\raggedright\strut
clearly described the question and why it's important 1; posted a
question 0.5; not raise any question 0.\strut
\end{minipage} & \begin{minipage}[t]{0.04\columnwidth}\raggedright\strut
X\strut
\end{minipage}\tabularnewline
\bottomrule
\end{longtable}

\section{Rubric for Response Paper}\label{rubric-for-response-paper}

\begin{longtable}[]{@{}lll@{}}
\toprule
\begin{minipage}[b]{0.15\columnwidth}\raggedright\strut
Item\strut
\end{minipage} & \begin{minipage}[b]{0.72\columnwidth}\raggedright\strut
Criteron\strut
\end{minipage} & \begin{minipage}[b]{0.05\columnwidth}\raggedright\strut
Grade\strut
\end{minipage}\tabularnewline
\midrule
\endhead
\begin{minipage}[t]{0.15\columnwidth}\raggedright\strut
Summary of the material\strut
\end{minipage} & \begin{minipage}[t]{0.72\columnwidth}\raggedright\strut
clearly described the logic and main arguments 2; covered the main
arguments 1; failed to capture the main arguments 0\strut
\end{minipage} & \begin{minipage}[t]{0.05\columnwidth}\raggedright\strut
X\strut
\end{minipage}\tabularnewline
\begin{minipage}[t]{0.15\columnwidth}\raggedright\strut
Learnt Point\strut
\end{minipage} & \begin{minipage}[t]{0.72\columnwidth}\raggedright\strut
clearly explained the learnt points and their importance 2; clearly
describe the points learnt 1; no learnt point mentioned 0\strut
\end{minipage} & \begin{minipage}[t]{0.05\columnwidth}\raggedright\strut
X\strut
\end{minipage}\tabularnewline
\begin{minipage}[t]{0.15\columnwidth}\raggedright\strut
Critical thinking\strut
\end{minipage} & \begin{minipage}[t]{0.72\columnwidth}\raggedright\strut
clearly opinions and why 2; have a perspective of the reading 1; no
perspective at all 0\strut
\end{minipage} & \begin{minipage}[t]{0.05\columnwidth}\raggedright\strut
X\strut
\end{minipage}\tabularnewline
\bottomrule
\end{longtable}

\clearpage

\section{Rubric for the Research
Proposal}\label{rubric-for-the-research-proposal}

\begin{longtable}[]{@{}lll@{}}
\toprule
\begin{minipage}[b]{0.12\columnwidth}\raggedright\strut
Item\strut
\end{minipage} & \begin{minipage}[b]{0.76\columnwidth}\raggedright\strut
Criteron\strut
\end{minipage} & \begin{minipage}[b]{0.03\columnwidth}\raggedright\strut
Grade\strut
\end{minipage}\tabularnewline
\midrule
\endhead
\begin{minipage}[t]{0.12\columnwidth}\raggedright\strut
Cover table (5\%)\strut
\end{minipage} & \begin{minipage}[t]{0.76\columnwidth}\raggedright\strut
Is the table fully filled? Is every element defined?\strut
\end{minipage} & \begin{minipage}[t]{0.03\columnwidth}\raggedright\strut
X\strut
\end{minipage}\tabularnewline
\begin{minipage}[t]{0.12\columnwidth}\raggedright\strut
Research Question Approval (5\%)\strut
\end{minipage} & \begin{minipage}[t]{0.76\columnwidth}\raggedright\strut
Was the research proposal approved by 2017-10-09?\strut
\end{minipage} & \begin{minipage}[t]{0.03\columnwidth}\raggedright\strut
X\strut
\end{minipage}\tabularnewline
\begin{minipage}[t]{0.12\columnwidth}\raggedright\strut
Brief of Theory (5\%)\strut
\end{minipage} & \begin{minipage}[t]{0.76\columnwidth}\raggedright\strut
Was the brief of the theory submitted by 2017-11-12?\strut
\end{minipage} & \begin{minipage}[t]{0.03\columnwidth}\raggedright\strut
X\strut
\end{minipage}\tabularnewline
\begin{minipage}[t]{0.12\columnwidth}\raggedright\strut
Introduction (10\%)\strut
\end{minipage} & \begin{minipage}[t]{0.76\columnwidth}\raggedright\strut
Is the research question well stated? Does the intro clearly explain the
importance of the study? Does the intro clearly explain the potential
contribution of this project?\strut
\end{minipage} & \begin{minipage}[t]{0.03\columnwidth}\raggedright\strut
X\strut
\end{minipage}\tabularnewline
\begin{minipage}[t]{0.12\columnwidth}\raggedright\strut
Literature Review (10\%)\strut
\end{minipage} & \begin{minipage}[t]{0.76\columnwidth}\raggedright\strut
Does the LR address more than three existing studies? Does the LR
clearly review the findings of the existing literature? Does the LR
clearly state how the existing literature serve as the basis of this
study?\strut
\end{minipage} & \begin{minipage}[t]{0.03\columnwidth}\raggedright\strut
X\strut
\end{minipage}\tabularnewline
\begin{minipage}[t]{0.12\columnwidth}\raggedright\strut
Theory (15\%)\strut
\end{minipage} & \begin{minipage}[t]{0.76\columnwidth}\raggedright\strut
Is the causal logic clearly stated? Are the concepts in the theory well
defined? Is the causal chain complete and consistent? Are the causal
inferences (hypotheses) clearly stated and consistent with the
theory?\strut
\end{minipage} & \begin{minipage}[t]{0.03\columnwidth}\raggedright\strut
X\strut
\end{minipage}\tabularnewline
\begin{minipage}[t]{0.12\columnwidth}\raggedright\strut
Research Design (20\%)\strut
\end{minipage} & \begin{minipage}[t]{0.76\columnwidth}\raggedright\strut
Does the author clearly describe the strategy to test the hypotheses?
Does the author well defend his/her method choice?\strut
\end{minipage} & \begin{minipage}[t]{0.03\columnwidth}\raggedright\strut
X\strut
\end{minipage}\tabularnewline
\begin{minipage}[t]{0.12\columnwidth}\raggedright\strut
Data (10\%)\strut
\end{minipage} & \begin{minipage}[t]{0.76\columnwidth}\raggedright\strut
Is there a complete plan of data collection? How do the data fit the
research design? Validities?\strut
\end{minipage} & \begin{minipage}[t]{0.03\columnwidth}\raggedright\strut
X\strut
\end{minipage}\tabularnewline
\begin{minipage}[t]{0.12\columnwidth}\raggedright\strut
Feasibility (5\%)\strut
\end{minipage} & \begin{minipage}[t]{0.76\columnwidth}\raggedright\strut
Is the research design a feaible one for a college student? What's the
potential difficulties the researcher may encounter?\strut
\end{minipage} & \begin{minipage}[t]{0.03\columnwidth}\raggedright\strut
X\strut
\end{minipage}\tabularnewline
\begin{minipage}[t]{0.12\columnwidth}\raggedright\strut
Citation (5\%)\strut
\end{minipage} & \begin{minipage}[t]{0.76\columnwidth}\raggedright\strut
Are the citations well presented? Is there a full bibliography attached?
Are the citation and bibliography styles consistent with the APSR
requirement?\strut
\end{minipage} & \begin{minipage}[t]{0.03\columnwidth}\raggedright\strut
X\strut
\end{minipage}\tabularnewline
\begin{minipage}[t]{0.12\columnwidth}\raggedright\strut
Overall writing (10\%)\strut
\end{minipage} & \begin{minipage}[t]{0.76\columnwidth}\raggedright\strut
Does the language well edited? Do the paragraphs well framed and
organized? Does the layout match the requirement?\strut
\end{minipage} & \begin{minipage}[t]{0.03\columnwidth}\raggedright\strut
X\strut
\end{minipage}\tabularnewline
\bottomrule
\end{longtable}

\clearpage

\section{CLAS Teaching Policies \& Resources --- Syllabus
Insert}\label{clas-teaching-policies-resources-syllabus-insert}

\subsection{Administrative Home}\label{administrative-home}

The College of Liberal Arts and Sciences is the administrative home of
this course and governs matters such as the add/drop deadlines, the
second-grade-only option, and other related issues. Different colleges
may have different policies. Questions may be addressed to 120 Schaeffer
Hall, or see the CLAS Academic Policies Handbook at
\url{https://clas.uiowa.edu/students/handbook}.

\subsection{Electronic Communication}\label{electronic-communication}

University policy specifies that students are responsible for all
official correspondences sent to their University of Iowa e-mail address
(({\textbf{???}})). Faculty and students should use this account for
correspondences
(\href{https://opsmanual.uiowa.edu/human-resources/professional-ethics-and-academic-responsibility\#15.2}{Operations
Manual, III.15.2}, k.11).

\subsection{Accommodations for
Disabilities}\label{accommodations-for-disabilities}

The University of Iowa is committed to providing an educational
experience that is accessible to all students. A student may request
academic accommodations for a disability (which includes but is not
limited to mental health, attention, learning, vision, and physical or
health-related conditions). A student seeking academic accommodations
should first register with Student Disability Services and then meet
with the course instructor privately in the instructor's office to make
particular arrangements. Reasonable accommodations are established
through an interactive process between the student, instructor, and SDS.
See \url{https://sds.studentlife.uiowa.edu/} for information.

\subsection{Nondiscrimination in the
Classroom}\label{nondiscrimination-in-the-classroom}

The University of Iowa is committed to making the classroom a respectful
and inclusive space for all people irrespective of their gender, sexual,
racial, religious or other identities. Toward this goal, students are
invited to optionally share their preferred names and pronouns with
their instructors and classmates. The University of Iowa prohibits
discrimination and harassment against individuals on the basis of race,
class, gender, sexual orientation, national origin, and other identity
categories set forth in the University's Human Rights policy. For more
information, contact the Office of Equal Opportunity and Diversity,
\href{mailto:diversity@uiowa.edu}{\nolinkurl{diversity@uiowa.edu}}, or
visit
\href{https://diversity.uiowa.edu/office/equal-opportunity-and-diversity}{diversity.uiowa.edu}.

\subsection{Academic Honesty}\label{academic-honesty}

All CLAS students or students taking classes offered by CLAS have, in
essence, agreed to the
\href{https://clas.uiowa.edu/students/handbook/academic-fraud-honor-code}{College's
Code of Academic Honesty}: ``I pledge to do my own academic work and to
excel to the best of my abilities, upholding the
\href{https://newstudents.uiowa.edu/iowa-challenge}{IOWA Challenge}. I
promise not to lie about my academic work, to cheat, or to steal the
words or ideas of others; nor will I help fellow students to violate the
Code of Academic Honesty.'' Any student committing academic misconduct
is reported to the College and placed on disciplinary probation or may
be suspended or expelled
(\href{https://clas.uiowa.edu/students/handbook}{CLAS Academic Policies
Handbook}).

\subsection{CLAS Final Examination
Policies}\label{clas-final-examination-policies}

The final examination schedule for each class is announced by the
Registrar generally by the fifth week of classes. Final exams are
offered only during the official final examination period. No exams of
any kind are allowed during the last week of classes. All students
should plan on being at the UI through the final examination period.
Once the Registrar has announced the date, time, and location of each
final exam, the complete schedule will be published on the Registrar's
web site and will be shared with instructors and students. It is the
student's responsibility to know the date, time, and place of a final
exam.

\subsection{Making a Suggestion or a
Complaint}\label{making-a-suggestion-or-a-complaint}

Students with a suggestion or complaint should first visit with the
instructor (and the course supervisor), and then with the departmental
DEO. Complaints must be made within six months of the incident
(\href{https://clas.uiowa.edu/students/handbook}{CLAS Academic Policies
Handbook}).

\subsection{Understanding Sexual
Harassment}\label{understanding-sexual-harassment}

Sexual harassment subverts the mission of the University and threatens
the well-being of students, faculty, and staff. All members of the UI
community have a responsibility to uphold this mission and to contribute
to a safe environment that enhances learning. Incidents of sexual
harassment should be reported immediately. See the UI
\href{https://osmrc.uiowa.edu/}{Office of the Sexual Misconduct Response
Coordinator} for assistance, definitions, and the full University
policy.

\subsection{Reacting Safely to Severe
Weather}\label{reacting-safely-to-severe-weather}

In severe weather, class members should seek appropriate shelter
immediately, leaving the classroom if necessary. The class will continue
if possible when the event is over. For more information on Hawk Alert
and the siren warning system, visit the
\href{https://police.uiowa.edu/emergency-communications}{Department of
Public Safety website}.

\clearpage

\section*{Reference}\label{reference}
\addcontentsline{toc}{section}{Reference}

\hypertarget{refs}{}
\hypertarget{ref-Babbie2012}{}
Babbie, Earl R. 2012. \emph{The Practice of Social Research}. 13th ed.
Australia: Wadsworth Cengage Learning.

\hypertarget{ref-VanEvera1997}{}
Van Evera, S. 1997. \emph{Guide to Methods for Students of Political
Science}. Cornell Paperbacks. Cornell University Press.



\clearpage
\end{document}

\makeatletter
\def\@maketitle{%
  \newpage
%  \null
%  \vskip 2em%
%  \begin{center}%
  \let \footnote \thanks
    {\fontsize{18}{20}\selectfont\raggedright  \setlength{\parindent}{0pt} \@title \par}%
}
%\fi
\makeatother
