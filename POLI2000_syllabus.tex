\documentclass[11pt,]{article}
\usepackage[margin=1in]{geometry}
\newcommand*{\authorfont}{\fontfamily{phv}\selectfont}
\usepackage{lmodern}
\usepackage{abstract}
\renewcommand{\abstractname}{}    % clear the title
\renewcommand{\absnamepos}{empty} % originally center
\newcommand{\blankline}{\quad\pagebreak[2]}

\providecommand{\tightlist}{%
  \setlength{\itemsep}{0pt}\setlength{\parskip}{0pt}} 
\usepackage{longtable,booktabs}

\usepackage{parskip}
\usepackage{titlesec}
\titlespacing\section{0pt}{12pt plus 4pt minus 2pt}{6pt plus 2pt minus 2pt}
\titlespacing\subsection{0pt}{12pt plus 4pt minus 2pt}{6pt plus 2pt minus 2pt}

\titleformat*{\subsubsection}{\normalsize\itshape}

\usepackage{titling}
\setlength{\droptitle}{-.25cm}

%\setlength{\parindent}{0pt}
%\setlength{\parskip}{6pt plus 2pt minus 1pt}
%\setlength{\emergencystretch}{3em}  % prevent overfull lines 

\usepackage[T1]{fontenc}
\usepackage[utf8]{inputenc}

\usepackage{fancyhdr}
\pagestyle{fancy}
\usepackage{lastpage}
\renewcommand{\headrulewidth}{0.3pt}
\renewcommand{\footrulewidth}{0.0pt} 
\lhead{}
\chead{}
\rhead{\footnotesize POLI 2000: Designing Political Research -- Fall 2017}
\lfoot{}
\cfoot{\small \thepage/\pageref*{LastPage}}
\rfoot{}

\fancypagestyle{firststyle}
{
\renewcommand{\headrulewidth}{0pt}%
   \fancyhf{}
   \fancyfoot[C]{\small \thepage/\pageref*{LastPage}}
}

%\def\labelitemi{--}
%\usepackage{enumitem}
%\setitemize[0]{leftmargin=25pt}
%\setenumerate[0]{leftmargin=25pt}




\makeatletter
\@ifpackageloaded{hyperref}{}{%
\ifxetex
  \usepackage[setpagesize=false, % page size defined by xetex
              unicode=false, % unicode breaks when used with xetex
              xetex]{hyperref}
\else
  \usepackage[unicode=true]{hyperref}
\fi
}
\@ifpackageloaded{color}{
    \PassOptionsToPackage{usenames,dvipsnames}{color}
}{%
    \usepackage[usenames,dvipsnames]{color}
}
\makeatother
\hypersetup{breaklinks=true,
            bookmarks=true,
            pdfauthor={ ()},
             pdfkeywords = {},  
            pdftitle={POLI 2000: Designing Political Research},
            colorlinks=true,
            citecolor=blue,
            urlcolor=blue,
            linkcolor=magenta,
            pdfborder={0 0 0}}
\urlstyle{same}  % don't use monospace font for urls


\setcounter{secnumdepth}{0}

\usepackage{longtable}




\usepackage{setspace}

\title{POLI 2000: Designing Political Research}
\author{Yue Hu}
\date{Fall 2017}


\begin{document}  

		\maketitle
		
	
		\thispagestyle{firststyle}

%	\thispagestyle{empty}


	\noindent \begin{tabular*}{\textwidth}{ @{\extracolsep{\fill}} lr @{\extracolsep{\fill}}}


E-mail: \texttt{\href{mailto:yue-hu-1@uiowa.edu}{\nolinkurl{yue-hu-1@uiowa.edu}}} & Web: TBD\\
Office Hours: 12:30 -- 15:30 M \& by Appointment  &  Class Hours: 15:30 -- 16:45 M/W\\
Office: 313 Shaeffer Hall  & Class Room: 105 EPB\\
	&  \\
	\hline
	\end{tabular*}
	
\vspace{2mm}
	


\section{Overview}\label{overview}

How do candidates win elections? Why do countries get involved in
international crises and wars? What makes a country more powerful than
the others? What explains the choices of violent non-state actors like
terrorists? There are just some of the questions that political
scientists study. The goal of this course is show you how to research
these questions as a scholar in political science. This course will
introduce students to political science research and various ways that
social scientific research is undertaken.

This class will help students comprehend the core elements to build a
political science research, such as concept, theory, hypothesis, and
evidence. You will also learn how to build new theories, develop
testable causal inferences, and design different approaches to examine
your theories empirically. Emphasis will be on an active hands-on
learning environment and fully interaction between the instructor and
students. Students can expect to understand the research produced in
Political Science and even other social scientific disciplines more
comprehensively.

\section{Requirements}\label{requirements}

I will base your grade for the course on your performance in the four
areas below. For each component of the course grade, I assign a
numerical score. I then calculate the course grade with the weighted
average of the component scores. Scores of 90-100 correspond to A, 80-89
to B, etc., with pluses and minuses for the top and bottom third of each
decile.

\begin{enumerate}
\def\labelenumi{\arabic{enumi}.}
\item
  Class attendance and performance (30\%: 10\% participation + 10\%
  leading discussion + 10\% attendance). Regarding participation, I am
  looking for you to show that you have read and critically evaluated
  the assigned readings and are engaged with our in-class discussions.
  Each student will
\item
  In-class quizzes (24\%)
\item
  Midterm examination (20\%)
\item
  Final examination (26\%)
\end{enumerate}

\section{Readings}\label{readings}

\subsection{Required textbook:}\label{required-textbook}

Earl R. Babbie. \emph{The Practice of Social Research}. 13th ed.
Australia: Wadsworth Cengage Learning, 2012. ISBN: 9781133049791
1133049796.

S. Van Evera. \emph{Guide to Methods for Students of Political Science}.
Cornell paperbacks. Cornell University Press, 1997. ISBN: 9780801484575.

\subsection{Week 1 (2017-08-21\textasciitilde{}2017-08-27): Being a
Political
Scientist}\label{week-1-2017-08-212017-08-27-being-a-political-scientist}

Robert O Keohane. ``Political Science as a Vocation''.
\emph{PS: Political Science and Politics} 42.02 (2009), pp.~359--363.

John S Dryzek. ``Revolutions without Enemies: Key Transformations in
Political Science''. \emph{American Political Science Review} 100.04
(2006), pp.~487--92.

Gary King. ``Publication, Publication''.
\emph{PS: Political Science and Politics} 39.01 (2006), pp.~119--125.

\subsection{Week 2 (2017-08-28\textasciitilde{}2017-09-03): Being
Scientific}\label{week-2-2017-08-282017-09-03-being-scientific}

Babbie (2012), pp.1-27, 112-120.

Gary King. ``Replication, Replication''.
\emph{PS: Political Science and Politics} 28.03 (1995), pp.~444--452.

Gabriel A Almond. ``Separate Tables: Schools and Sects in Political
Science''. \emph{PS: Political Science and Politics} 21.4 (1988),
pp.~828--42.

\emph{Recommended:}

Imre Lakatos and Musgrave Alan. ``Falsification and the Methodology of
Scientific Research Programmes''.
\emph{Criticism and the Growth of Knowledge} (1970), pp.~91--180.

\subsection{Week 3 (2017-09-04\textasciitilde{}2017-09-10): What's A
Good Question (Labor
Day)}\label{week-3-2017-09-042017-09-10-whats-a-good-question-labor-day}

Barbara Geddes. ``Big Questions, Little Answers: How the Questions You
Choose Affect the Answer You Get''. In:
\emph{Paradigms and Sand Castles: Theory Building and Research Design in Comparative Politics}.
Ann Arbor: University of Michigan Press, 2010. Chap. 2, pp.~27--88.

Van Evera (1997), pp.97-99.

\subsection{Week 4 (2017-09-11\textasciitilde{}2017-09-17): How to Find
Research
Question}\label{week-4-2017-09-112017-09-17-how-to-find-research-question}

Babbie (2012), pp.91-112.

Efren O Perez and Margit Tavits. ``Language Shapes People's Time
Perspective and Support for Future-Oriented Policies''.
\emph{American Journal of Political Science} (2017), pp.~1--13.

Timothy J McKeown. ``Case Studies and the Statistical Worldview Review
of King, Keohane, and Verba's Designing Social Inquiry Scientific
Inference in Qualitative Research''. \emph{International organization}
53.01 (1999), pp.~161--190.

Charles C Ragin and Lisa M Amoroso.
\emph{Constructing Social Research: The Unity and Diversity of Method}.
Pine Forge Press, 2010. (Chapter 1, 2)

\subsection{Week 5 (2017-09-18\textasciitilde{}2017-09-24):
Concepts}\label{week-5-2017-09-182017-09-24-concepts}

Babbie (2012), pp.165-177.

Michael Barnett and Raymond Duvall. ``Power in International Politics''.
\emph{International Organization} 59.01 (2005), pp.~39--75.

David Collier and Steven Levitsky. ``Democracy with Adjectives:
Conceptual Innovation in Comparative Research''. \emph{World Politics}
49.03 (1997), pp.~430--451.

David Collier and James E Mahon. ``Conceptual ``Stretching'' Revisited:
Adapting Categories in Comparative Analysis''.
\emph{American Political Science Review} 87.04 (1993), pp.~845--855.

Giovanni Sartori. ``Concept misformation in comparative politics''.
\emph{American political science review} 64.04 (1970), pp.~1033--1053.

\emph{Recommended:}

Peter Bachrach and Morton S Baratz. ``Two Faces of Power''.
\emph{American Political Science Review} 56.04 (1962), pp.~947--952.

\subsection{Week 6 (2017-09-25\textasciitilde{}2017-10-01):
Measurement}\label{week-6-2017-09-252017-10-01-measurement}

Babbie (2012), pp.177-194, 197-223; 57-83.

Jason Seawright and David Collier. ``Rival Strategies of Validation
Tools for Evaluating Measures of Democracy''.
\emph{Comparative Political Studies} 47.1 (2014), pp.~111--138.

John Gerring. ``Causal mechanisms: Yes, But\ldots{}''
\emph{Comparative Political Studies} 43.11 (2010), pp.~1499--526.

Robert Adcock and David Collier. ``Measurement Validity: A Shared
Standard for Qualitative and Quantitative Research''.
\emph{American Political Science Review} 33 (2001), pp.~529--546.

\subsection{Week 7 (2017-10-02\textasciitilde{}2017-10-08):
Theory}\label{week-7-2017-10-022017-10-08-theory}

Van Evera (1997), pp.7-50.

Midterm Review

\subsection{Week 8 (2017-10-09\textasciitilde{}2017-10-15): Experimental
Logic and
Design}\label{week-8-2017-10-092017-10-15-experimental-logic-and-design}

Midterm.

Babbie (2012), pp.271-291.

Alan S Gerber and Donald P Green. ``Field Experiments and Natural
Experiments''. In: \emph{The Oxford Handbook of Political Science}. Ed.
by Robert E. Goodin. 2011.
\url{http://www.oxfordhandbooks.com/view/10.1093/oxfordhb/9780199604456.001.0001/oxfordhb-9780199604456-e-050?mediaType=Article}
(visited on 06/15/2017).

Alex Mintz. ``Foreign Policy Decision Making in Familiar and Unfamiliar
Settings''. \emph{Journal of Conflict Resolution} 48.1 (2004),
pp.~91--104.

\subsection{Week 9 (2017-10-16\textasciitilde{}2017-10-22): Principles
of Case
Study}\label{week-9-2017-10-162017-10-22-principles-of-case-study}

Jack S Levy. ``Case Studies: Types, Designs, and Logics of Inference''.
\emph{Conflict Management and Peace Science} 25.1 (2008), pp.~1--18.

John Gerring. ``What is a Case Study and What is it Good for?''
\emph{American Political Science Review} 98.02 (2004), pp.~341--354.

Timothy J McKeown. ``Case Studies and the Limits of the Quantitative
Worldview''. In:
\emph{Rethinking Social Inquiry: Diverse Tools, Shared Standards}. Ed.
by David Collier and Henry E. Brady. Lanham, MD: Rowman and Littlefield,
2004, pp.~139--167.

Adam Przeworski and Henry Teune.
\emph{The Logic of Comparative Social Inquiry}. New York: Joh Wiley and
Sons, 1970. 31-39, 74-87.

\subsection{Week 10 (2017-10-23\textasciitilde{}2017-10-29): Case Study
in Practice}\label{week-10-2017-10-232017-10-29-case-study-in-practice}

Van Evera (1997), pp.49-88.

Barbara Geddes. ``How the Cases You Choose Affect the Answers You Get:
Selection Bias in Comparative Politics''. In:
\emph{Paradigms and Sand Castles: Theory Building and Research Design in Comparative Politics}.
Ann Arbor: University of Michigan Press, 2010, pp.~89--129.

Gerardo L. Munck. ``Tools for Qualitative Research''. In:
\emph{Rethinking Social Inquiry: Diverse Tools, Shared Standards}. Ed.
by David Collier and Henry E. Brady. Lanham, MD: Rowman and Littlefield,
2004, pp.~105--121.

Benjamin A Most and Harvey Starr. ``Case Selection, Conceptualizations
and Basic Logic in the Study of War''.
\emph{American Journal of Political Science} (1982), pp.~834--856.

\subsection{Week 11 (2017-10-30\textasciitilde{}2017-11-05): A Glance of
Other Small-N
Methods}\label{week-11-2017-10-302017-11-05-a-glance-of-other-small-n-methods}

Babbie (2012), pp.295-321.

Andrew Bennett. ``Process Tracing: A Bayesian Perspective''. In:
\emph{Oxford Handbook of Political Methodology}. Ed. by Janet Box
Steffensmeier, Henry Brady and David Coiller. 2008, pp.~702--21.

Giovanni Capoccia and R Daniel Kelemen. ``The Study of Critical
Junctures: Theory, Narrative, and Counterfactuals in Historical
Institutionalism''. \emph{World Politics} 59.03 (2007), pp.~341--369.

Clifford Geertz. ``Thick Description: Toward an Interpretive Theory of
Culture''. \emph{Readings in the Philosophy of Social Science} (1994),
pp.~213--31.

\subsection{Week 12 (2017-11-06\textasciitilde{}2017-11-12):
Understanding Large-N
Analyses}\label{week-12-2017-11-062017-11-12-understanding-large-n-analyses}

Babbie (2012), pp.415-438.

Wenfang Tang, Yue Hu and Shuai Jin. ``Affirmative Inaction: Language
Education and Labor Mobility among China's Muslim Minorities''.
\emph{Chinese Sociological Review} (4 2016), pp.~346--66.

Emilie M Hafner-Burton and Alexander H Montgomery. ``Power Positions:
International Organizations, Social Networks, and Conflict''.
\emph{Journal of Conflict Resolution} 50.1 (2006), pp.~3--27.

James Mahoney and Gary Goertz. ``A Tale of Two Cultures: Contrasting
Quantitative and Qualitative Research''. \emph{Political Analysis} 14.3
(2006), pp.~227--49.

\subsection{Week 13 (2017-11-13\textasciitilde{}2017-11-19):
Professionalization}\label{week-13-2017-11-132017-11-19-professionalization}

Babbie (2012), pp.498-519.

Van Evera (1997), pp.99-111.

\href{https://www.youtube.com/watch?v=bwNBXuz2eRg}{APSA 2014: Policy
Bargaining and International Conflict}

\href{https://www.youtube.com/watch?v=Z4ISkF2H4tk}{MPSA 2017: Trump
Scenes}

\subsection{Week 14 (2017-11-20\textasciitilde{}2017-11-26): Thanks
Giving Week}\label{week-14-2017-11-202017-11-26-thanks-giving-week}

\subsection{Week 15 (2017-11-27\textasciitilde{}2017-12-03):
Presentation}\label{week-15-2017-11-272017-12-03-presentation}

\subsection{Week 16 (2017-12-04\textasciitilde{}2017-12-10):
Presentation}\label{week-16-2017-12-042017-12-10-presentation}

\subsection*{Week 17 (2017-12-11\textasciitilde{}2017-12-17): Final
Week}\label{week-17-2017-12-112017-12-17-final-week}
\addcontentsline{toc}{subsection}{Week 17
(2017-12-11\textasciitilde{}2017-12-17): Final Week}

\hypertarget{refs}{}
\hypertarget{ref-Babbie2012}{}
Babbie, Earl R. 2012. \emph{The Practice of Social Research}. 13th ed.
Australia: Wadsworth Cengage Learning.

\hypertarget{ref-VanEvera1997}{}
Van Evera, S. 1997. \emph{Guide to Methods for Students of Political
Science}. Cornell Paperbacks. Cornell University Press.




\end{document}

\makeatletter
\def\@maketitle{%
  \newpage
%  \null
%  \vskip 2em%
%  \begin{center}%
  \let \footnote \thanks
    {\fontsize{18}{20}\selectfont\raggedright  \setlength{\parindent}{0pt} \@title \par}%
}
%\fi
\makeatother
