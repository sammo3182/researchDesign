\documentclass[11pt,]{article}
\usepackage[margin=1in]{geometry}
\newcommand*{\authorfont}{\fontfamily{phv}\selectfont}
\usepackage{lmodern}
\usepackage{abstract}
\renewcommand{\abstractname}{}    % clear the title
\renewcommand{\absnamepos}{empty} % originally center
\newcommand{\blankline}{\quad\pagebreak[2]}

\providecommand{\tightlist}{%
  \setlength{\itemsep}{0pt}\setlength{\parskip}{0pt}} 
\usepackage{longtable,booktabs}

\usepackage{parskip}
\usepackage{titlesec}
\titlespacing\section{0pt}{12pt plus 4pt minus 2pt}{6pt plus 2pt minus 2pt}
\titlespacing\subsection{0pt}{12pt plus 4pt minus 2pt}{6pt plus 2pt minus 2pt}

\titleformat*{\subsubsection}{\normalsize\itshape}

\usepackage{titling}
\setlength{\droptitle}{-.25cm}

%\setlength{\parindent}{0pt}
%\setlength{\parskip}{6pt plus 2pt minus 1pt}
%\setlength{\emergencystretch}{3em}  % prevent overfull lines 

\usepackage[T1]{fontenc}
\usepackage[utf8]{inputenc}

\usepackage{fancyhdr}
\pagestyle{fancy}
\usepackage{lastpage}
\renewcommand{\headrulewidth}{0.3pt}
\renewcommand{\footrulewidth}{0.0pt} 
\lhead{}
\chead{}
\rhead{\footnotesize POLI 2000: Designing Political Research -- Fall 2017}
\lfoot{}
\cfoot{\small \thepage/\pageref*{LastPage}}
\rfoot{}

\fancypagestyle{firststyle}
{
\renewcommand{\headrulewidth}{0pt}%
   \fancyhf{}
   \fancyfoot[C]{\small \thepage/\pageref*{LastPage}}
}

%\def\labelitemi{--}
%\usepackage{enumitem}
%\setitemize[0]{leftmargin=25pt}
%\setenumerate[0]{leftmargin=25pt}




\makeatletter
\@ifpackageloaded{hyperref}{}{%
\ifxetex
  \usepackage[setpagesize=false, % page size defined by xetex
              unicode=false, % unicode breaks when used with xetex
              xetex]{hyperref}
\else
  \usepackage[unicode=true]{hyperref}
\fi
}
\@ifpackageloaded{color}{
    \PassOptionsToPackage{usenames,dvipsnames}{color}
}{%
    \usepackage[usenames,dvipsnames]{color}
}
\makeatother
\hypersetup{breaklinks=true,
            bookmarks=true,
            pdfauthor={ ()},
             pdfkeywords = {},  
            pdftitle={POLI 2000: Designing Political Research},
            colorlinks=true,
            citecolor=blue,
            urlcolor=blue,
            linkcolor=magenta,
            pdfborder={0 0 0}}
\urlstyle{same}  % don't use monospace font for urls


\setcounter{secnumdepth}{0}

\usepackage{longtable}




\usepackage{setspace}

\title{POLI 2000: Designing Political Research}
\author{Yue Hu}
\date{Fall 2017}


\usepackage{amsthm}
\newtheorem{theorem}{Theorem}[section]
\newtheorem{lemma}{Lemma}[section]
\theoremstyle{definition}
\newtheorem{definition}{Definition}[section]
\newtheorem{corollary}{Corollary}[section]
\newtheorem{proposition}{Proposition}[section]
\theoremstyle{definition}
\newtheorem{example}{Example}[section]
\theoremstyle{remark}
\newtheorem*{remark}{Remark}
\begin{document}  

		\maketitle
		
	
		\thispagestyle{firststyle}

%	\thispagestyle{empty}


	\noindent \begin{tabular*}{\textwidth}{ @{\extracolsep{\fill}} lr @{\extracolsep{\fill}}}


E-mail: \texttt{\href{mailto:yue-hu-1@uiowa.edu}{\nolinkurl{yue-hu-1@uiowa.edu}}} & Web: TBD\\
Office Hours: 12:30 -- 15:30 M \& by Appointment  &  Class Hours: 15:30 -- 16:45 M/W\\
Office: 313 Shaeffer Hall  & Class Room: 105 EPB\\
	&  \\
	\hline
	\end{tabular*}
	
\vspace{2mm}
	


\section{Overview}\label{overview}

How do candidates win elections? Why do countries get involved in
international crises and wars? What makes a country more powerful than
the others? What explains the choices of violent non-state actors like
terrorists? There are just some of the questions that political
scientists study. The goal of this course is show you how to research
these questions as a scholar in political science. This course will
introduce students to political science research and various ways that
social scientific research is undertaken.

This class will help students comprehend the core elements to build a
political science research, such as concept, theory, hypothesis, and
evidence. You will also learn how to build new theories, develop
testable causal inferences, and design different approaches to examine
your theories empirically. Emphasis will be on an active hands-on
learning environment and fully interaction between the instructor and
students. Students can expect to understand the research produced in
Political Science and even other social scientific disciplines more
comprehensively.

\section{Requirements}\label{requirements}

I will base your grade for the course on your performance in the four
areas below. You will get a score for each component, and your total
grade will be the sum of these scores. The grade points will be
translated to letter grades in the following way: 93-100 A, 90-92.9 A-,
87-89.9 B+, 83-86.9 B, 80-82.9 B-, 77-79.9 C+, 73-76.9 C, 70-72.9 C-,
67-69.9 D+, 63-66.9 D, 60-62.3 D-, 59.9 or less F.

\subsection{Class attendance and performance (25\%: 5\% attendance +
10\% participation + 10\% opening
presentation).}\label{class-attendance-and-performance-25-5-attendance-10-participation-10-opening-presentation.}

Regarding participation, I am looking for you to show that you have read
and critically evaluated the assigned readings and are engaged with our
in-class discussions. Before each week's class you will receive a
one-page reading guideline. The guideline will provide a recommanded
reading order and several questions to help you comprehend the
materials.

In the first two week, you decide which week you are interested to
present and report to the instructor. Be aware: there is a limited
number of student to present in each week. So, if your favoriate week is
full, you have to choose some other week instead. Starting from Week 3,
the first 20 minutes of each class will be students' show time. You will
be required to work as a team with the ones who also sign in this week.
You and your partners will need to prepare a 10-min presentation about
one of the \emph{Recommanded} readings of the given week. You will
introduce us what the mateiral talks about and, more importantly, why it
appears in this week and how it relates to the required readings. I will
evaluate your performance according to the rubric attached at the end of
this syllabus. The socre will be identical for every member of the team.

\subsection{In-class quizzes (10\%)}\label{in-class-quizzes-10}

You will get ten pop quizzes throughout the semester. Each quiz includes
2-3 questions---all about the required readings of the given week.
(Hint: some questions may come from the reading guide of the week.) At
the end of the semester, the eight highest scores of the ten quizzes
will be recorded to contribute to your final grade. Quizzes will be
administered and completed either at the beginning or the end of class.
You cannot make up any missed or failed quizzes for any reason. Feel
free to discuss the readings with your classmates prior to class.
\emph{However, you cannot share your answers to the reading questions
with your classmates.}

\subsection{Critical Response paper
(12\%)}\label{critical-response-paper-12}

You are expected to submit response papers for the readings of three
\emph{weeks}. You are free to pick any week you like before we start
discussing the materials. (You are free to write a paper for the week
your group lead the opening presentation. However, each group member
need to write his/her response paper independently.) That is, the
response paper about a given week's reading is due by the BEGINNING of
the Monday class (i.e., 15:30) in that week (submit to ICON by this
time). Papers will be evaluate in a 1--100 scale following the rubric
attached at the end. Late paper will be penalized 10 points for each day
of being late. Each paper should be 2-4 pages, double-spaced, 1 inch
margins, and in 12 font size.

In the paper, You needs to cover at three two reading materials of the
week. Moreover, I would like to see

\begin{enumerate}
\def\labelenumi{\arabic{enumi}.}
\tightlist
\item
  One and ONLY one paragraphs to summarize each material.
\item
  Some discussion about what do you learn from these materials about the
  week topic.
\item
  Your perspective about the argument in the materials---do you agree
  with the author? If yes, why are his/her arguments convincing for you?
  If not, what's your opinion of the topic?
\item
  Cite other sources if appropriate (but not required).
\end{enumerate}

\subsection{Examination (20\%)}\label{examination-20}

There is only one exam (viz. midterm) for this course. The exam is
comprised of identification and short essay questions. The exam will be
held on 2017-10-09 in the same class of the lecture.

\subsection{Research Proposal (23\%)}\label{research-proposal-23}

You are expected to submit a research proposal at the end of the
semester. It could be a proposal for your graduation thesis or for a
funding application. The proposal should include at least six parts:

\begin{enumerate}
\def\labelenumi{\arabic{enumi}.}
\tightlist
\item
  A overview table (see template on ICON).
\item
  An introduction of your research question and why it is important.
\item
  A brief literature review about what scholars have done on this topic.
  You need to cite at least three articles published in academic
  journals of political science. Your citation should be in the style
  used by the American Political Science Association (see the guideline
  \href{http://www.apsanet.org/portals/54/Files/Publications/APSAStyleManual2006.pdf}{here}).
\item
  A section discussing your theory and hypotheses.
\item
  A section discussing the data and method you plan to use to test your
  theory and why they are the best choice for your project.
\item
  A section discussing the operational feasibility of your research
  design.
\end{enumerate}

The proposal is expected at least 10 pages (excluding the title and
reference pages), double-spaced, 1 inch margins, and in 12 font size.
The proposal will be evaluated according to the rubric attached.

Here are some important dates relating to the research proposal:

\begin{itemize}
\tightlist
\item
  You must get your research question approved by the instructor by
  2017-10-01.
\item
  You must submit a brief about your theory by 2017-11-12.
\item
  The proposal is due by 2017-12-15. Late paper will be penalized 30
  points for each day of being late.
\end{itemize}

\subsubsection{Extra credit}\label{extra-credit}

You will get 5 extra credits if you use bibliography management
softwares (Endnote, Jabref, Zotero, etc. see more information about
these softwares
\href{https://en.wikipedia.org/wiki/Comparison_of_reference_management_software}{here}.)
and submit relevant bibliography files with your proposal.

You can get another 5 credits by attending an academic workshop held by
the Department of Political Science and send me a one-paragraph
discussion about what the workshop is, how you feel about it, and a
selfie at the workshop location.

\section{Required textbook:}\label{required-textbook}

Earl R. Babbie. \emph{The Practice of Social Research}. 13th ed.
Australia: Wadsworth Cengage Learning, 2012. ISBN: 9781133049791
1133049796.

S. Van Evera. \emph{Guide to Methods for Students of Political Science}.
Cornell paperbacks. Cornell University Press, 1997. ISBN: 9780801484575.

\section{Schedule}\label{schedule}

\subsection{Week 1 (2017-08-21\textasciitilde{}2017-08-27): Being a
Political
Scientist}\label{week-1-2017-08-212017-08-27-being-a-political-scientist}

Robert O Keohane. ``Political Science as a Vocation''.
\emph{PS: Political Science and Politics} 42.02 (2009), pp.~359--363.

John S Dryzek. ``Revolutions without Enemies: Key Transformations in
Political Science''. \emph{American Political Science Review} 100.04
(2006), pp.~487--92.

Gary King. ``Publication, Publication''.
\emph{PS: Political Science and Politics} 39.01 (2006), pp.~119--125.

\subsection{Week 2 (2017-08-28\textasciitilde{}2017-09-03): Being
Scientific}\label{week-2-2017-08-282017-09-03-being-scientific}

Babbie (2012), pp.1-27, 112-120.

Gary King. ``Replication, Replication''.
\emph{PS: Political Science and Politics} 28.03 (1995), pp.~444--452.

Gabriel A Almond. ``Separate Tables: Schools and Sects in Political
Science''. \emph{PS: Political Science and Politics} 21.4 (1988),
pp.~828--42.

\emph{Recommended:}

Imre Lakatos and Musgrave Alan. ``Falsification and the Methodology of
Scientific Research Programmes''.
\emph{Criticism and the Growth of Knowledge} (1970), pp.~91--180.

\subsection{Week 3 (2017-09-04\textasciitilde{}2017-09-10): What's A
Good Question (Labor
Day)}\label{week-3-2017-09-042017-09-10-whats-a-good-question-labor-day}

Barbara Geddes. ``Big Questions, Little Answers: How the Questions You
Choose Affect the Answer You Get''. In:
\emph{Paradigms and Sand Castles: Theory Building and Research Design in Comparative Politics}.
Ann Arbor: University of Michigan Press, 2010. Chap. 2, pp.~27--88.

Van Evera (1997), pp.97-99.

\subsection{Week 4 (2017-09-11\textasciitilde{}2017-09-17): How to Find
Research
Question}\label{week-4-2017-09-112017-09-17-how-to-find-research-question}

Babbie (2012), pp.91-112.

Efren O Perez and Margit Tavits. ``Language Shapes People's Time
Perspective and Support for Future-Oriented Policies''.
\emph{American Journal of Political Science} (2017), pp.~1--13.

Timothy J McKeown. ``Case Studies and the Statistical Worldview Review
of King, Keohane, and Verba's Designing Social Inquiry Scientific
Inference in Qualitative Research''. \emph{International organization}
53.01 (1999), pp.~161--190.

Charles C Ragin and Lisa M Amoroso.
\emph{Constructing Social Research: The Unity and Diversity of Method}.
Pine Forge Press, 2010. (Chapter 1, 2)

\subsection{Week 5 (2017-09-18\textasciitilde{}2017-09-24):
Concepts}\label{week-5-2017-09-182017-09-24-concepts}

Babbie (2012), pp.165-177.

Michael Barnett and Raymond Duvall. ``Power in International Politics''.
\emph{International Organization} 59.01 (2005), pp.~39--75.

David Collier and Steven Levitsky. ``Democracy with Adjectives:
Conceptual Innovation in Comparative Research''. \emph{World Politics}
49.03 (1997), pp.~430--451.

David Collier and James E Mahon. ``Conceptual ``Stretching'' Revisited:
Adapting Categories in Comparative Analysis''.
\emph{American Political Science Review} 87.04 (1993), pp.~845--855.

Giovanni Sartori. ``Concept misformation in comparative politics''.
\emph{American political science review} 64.04 (1970), pp.~1033--1053.

\emph{Recommended:}

Peter Bachrach and Morton S Baratz. ``Two Faces of Power''.
\emph{American Political Science Review} 56.04 (1962), pp.~947--952.

\subsection{Week 6 (2017-09-25\textasciitilde{}2017-10-01):
Measurement}\label{week-6-2017-09-252017-10-01-measurement}

Babbie (2012), pp.177-194, 197-223; 57-83.

Jason Seawright and David Collier. ``Rival Strategies of Validation
Tools for Evaluating Measures of Democracy''.
\emph{Comparative Political Studies} 47.1 (2014), pp.~111--138.

John Gerring. ``Causal mechanisms: Yes, But\ldots{}''
\emph{Comparative Political Studies} 43.11 (2010), pp.~1499--526.

Robert Adcock and David Collier. ``Measurement Validity: A Shared
Standard for Qualitative and Quantitative Research''.
\emph{American Political Science Review} 33 (2001), pp.~529--546.

\subsection{Week 7 (2017-10-02\textasciitilde{}2017-10-08): Measurement
in Practice}\label{week-7-2017-10-022017-10-08-measurement-in-practice}

Van Evera (1997), pp.7-50.

Midterm Review

\subsection{Week 8 (2017-10-09\textasciitilde{}2017-10-15): What's
Theory?}\label{week-8-2017-10-092017-10-15-whats-theory}

Midterm.

\subsection{Week 9 (2017-10-16\textasciitilde{}2017-10-22): Theory and
Causal
Inference}\label{week-9-2017-10-162017-10-22-theory-and-causal-inference}

\subsection{Week 10 (2017-10-23\textasciitilde{}2017-10-29):
Experimental Logic and
Design}\label{week-10-2017-10-232017-10-29-experimental-logic-and-design}

Babbie (2012), pp.271-291.

Alan S Gerber and Donald P Green. ``Field Experiments and Natural
Experiments''. In: \emph{The Oxford Handbook of Political Science}. Ed.
by Robert E. Goodin. 2011.
\url{http://www.oxfordhandbooks.com/view/10.1093/oxfordhb/9780199604456.001.0001/oxfordhb-9780199604456-e-050?mediaType=Article}
(visited on 06/15/2017).

Alex Mintz. ``Foreign Policy Decision Making in Familiar and Unfamiliar
Settings''. \emph{Journal of Conflict Resolution} 48.1 (2004),
pp.~91--104.

\subsection{Week 11 (2017-10-30\textasciitilde{}2017-11-05): Principles
of Case
Study}\label{week-11-2017-10-302017-11-05-principles-of-case-study}

Jack S Levy. ``Case Studies: Types, Designs, and Logics of Inference''.
\emph{Conflict Management and Peace Science} 25.1 (2008), pp.~1--18.

John Gerring. ``What is a Case Study and What is it Good for?''
\emph{American Political Science Review} 98.02 (2004), pp.~341--354.

Timothy J McKeown. ``Case Studies and the Limits of the Quantitative
Worldview''. In:
\emph{Rethinking Social Inquiry: Diverse Tools, Shared Standards}. Ed.
by David Collier and Henry E. Brady. Lanham, MD: Rowman and Littlefield,
2004, pp.~139--167.

Adam Przeworski and Henry Teune.
\emph{The Logic of Comparative Social Inquiry}. New York: Joh Wiley and
Sons, 1970. 31-39, 74-87.

\subsection{Week 12 (2017-11-06\textasciitilde{}2017-11-12): Case Study
in Practice}\label{week-12-2017-11-062017-11-12-case-study-in-practice}

Van Evera (1997), pp.49-88.

Barbara Geddes. ``How the Cases You Choose Affect the Answers You Get:
Selection Bias in Comparative Politics''. In:
\emph{Paradigms and Sand Castles: Theory Building and Research Design in Comparative Politics}.
Ann Arbor: University of Michigan Press, 2010, pp.~89--129.

Gerardo L. Munck. ``Tools for Qualitative Research''. In:
\emph{Rethinking Social Inquiry: Diverse Tools, Shared Standards}. Ed.
by David Collier and Henry E. Brady. Lanham, MD: Rowman and Littlefield,
2004, pp.~105--121.

Benjamin A Most and Harvey Starr. ``Case Selection, Conceptualizations
and Basic Logic in the Study of War''.
\emph{American Journal of Political Science} (1982), pp.~834--856.

\subsection{Week 13 (2017-11-13\textasciitilde{}2017-11-19): A Glance of
Other Small-N
Methods}\label{week-13-2017-11-132017-11-19-a-glance-of-other-small-n-methods}

Babbie (2012), pp.295-321.

Andrew Bennett. ``Process Tracing: A Bayesian Perspective''. In:
\emph{Oxford Handbook of Political Methodology}. Ed. by Janet Box
Steffensmeier, Henry Brady and David Coiller. 2008, pp.~702--21.

Giovanni Capoccia and R Daniel Kelemen. ``The Study of Critical
Junctures: Theory, Narrative, and Counterfactuals in Historical
Institutionalism''. \emph{World Politics} 59.03 (2007), pp.~341--369.

Clifford Geertz. ``Thick Description: Toward an Interpretive Theory of
Culture''. \emph{Readings in the Philosophy of Social Science} (1994),
pp.~213--31.

\subsection{Week 14 (2017-11-20\textasciitilde{}2017-11-26): Thanks
Giving Week}\label{week-14-2017-11-202017-11-26-thanks-giving-week}

\subsection{Week 15 (2017-11-27\textasciitilde{}2017-12-03):
Understanding Large-N
Analyses}\label{week-15-2017-11-272017-12-03-understanding-large-n-analyses}

Babbie (2012), pp.415-438.

Wenfang Tang, Yue Hu and Shuai Jin. ``Affirmative Inaction: Language
Education and Labor Mobility among China's Muslim Minorities''.
\emph{Chinese Sociological Review} (4 2016), pp.~346--66.

Emilie M Hafner-Burton and Alexander H Montgomery. ``Power Positions:
International Organizations, Social Networks, and Conflict''.
\emph{Journal of Conflict Resolution} 50.1 (2006), pp.~3--27.

James Mahoney and Gary Goertz. ``A Tale of Two Cultures: Contrasting
Quantitative and Qualitative Research''. \emph{Political Analysis} 14.3
(2006), pp.~227--49.

\subsection{Week 16 (2017-12-04\textasciitilde{}2017-12-10):
Professionalization}\label{week-16-2017-12-042017-12-10-professionalization}

Babbie (2012), pp.498-519.

Van Evera (1997), pp.99-111.

\href{https://www.youtube.com/watch?v=bwNBXuz2eRg}{APSA 2014: Policy
Bargaining and International Conflict}

\href{https://www.youtube.com/watch?v=Z4ISkF2H4tk}{MPSA 2017: Trump
Scenes}

\subsection{Week 17 (2017-12-11\textasciitilde{}2017-12-17): Final
Week}\label{week-17-2017-12-112017-12-17-final-week}

\clearpage

\section{CLAS Teaching Policies \& Resources --- Syllabus
Insert}\label{clas-teaching-policies-resources-syllabus-insert}

\subsection{Administrative Home}\label{administrative-home}

The College of Liberal Arts and Sciences is the administrative home of
this course and governs matters such as the add/drop deadlines, the
second-grade-only option, and other related issues. Different colleges
may have different policies. Questions may be addressed to 120 Schaeffer
Hall, or see the CLAS Academic Policies Handbook at
\url{https://clas.uiowa.edu/students/handbook}.

\subsection{Electronic Communication}\label{electronic-communication}

University policy specifies that students are responsible for all
official correspondences sent to their University of Iowa e-mail address
(({\textbf{???}})). Faculty and students should use this account for
correspondences
(\href{https://opsmanual.uiowa.edu/human-resources/professional-ethics-and-academic-responsibility\#15.2}{Operations
Manual, III.15.2}, k.11).

\subsection{Accommodations for
Disabilities}\label{accommodations-for-disabilities}

The University of Iowa is committed to providing an educational
experience that is accessible to all students. A student may request
academic accommodations for a disability (which includes but is not
limited to mental health, attention, learning, vision, and physical or
health-related conditions). A student seeking academic accommodations
should first register with Student Disability Services and then meet
with the course instructor privately in the instructor's office to make
particular arrangements. Reasonable accommodations are established
through an interactive process between the student, instructor, and SDS.
See \url{https://sds.studentlife.uiowa.edu/} for information.

\subsection{Nondiscrimination in the
Classroom}\label{nondiscrimination-in-the-classroom}

The University of Iowa is committed to making the classroom a respectful
and inclusive space for all people irrespective of their gender, sexual,
racial, religious or other identities. Toward this goal, students are
invited to optionally share their preferred names and pronouns with
their instructors and classmates. The University of Iowa prohibits
discrimination and harassment against individuals on the basis of race,
class, gender, sexual orientation, national origin, and other identity
categories set forth in the University's Human Rights policy. For more
information, contact the Office of Equal Opportunity and Diversity,
\href{mailto:diversity@uiowa.edu}{\nolinkurl{diversity@uiowa.edu}}, or
visit
\href{https://diversity.uiowa.edu/office/equal-opportunity-and-diversity}{diversity.uiowa.edu}.

\subsection{Academic Honesty}\label{academic-honesty}

All CLAS students or students taking classes offered by CLAS have, in
essence, agreed to the
\href{https://clas.uiowa.edu/students/handbook/academic-fraud-honor-code}{College's
Code of Academic Honesty}: ``I pledge to do my own academic work and to
excel to the best of my abilities, upholding the
\href{https://newstudents.uiowa.edu/iowa-challenge}{IOWA Challenge}. I
promise not to lie about my academic work, to cheat, or to steal the
words or ideas of others; nor will I help fellow students to violate the
Code of Academic Honesty.'' Any student committing academic misconduct
is reported to the College and placed on disciplinary probation or may
be suspended or expelled
(\href{https://clas.uiowa.edu/students/handbook}{CLAS Academic Policies
Handbook}).

\subsection{CLAS Final Examination
Policies}\label{clas-final-examination-policies}

The final examination schedule for each class is announced by the
Registrar generally by the fifth week of classes. Final exams are
offered only during the official final examination period. No exams of
any kind are allowed during the last week of classes. All students
should plan on being at the UI through the final examination period.
Once the Registrar has announced the date, time, and location of each
final exam, the complete schedule will be published on the Registrar's
web site and will be shared with instructors and students. It is the
student's responsibility to know the date, time, and place of a final
exam.

\subsection{Making a Suggestion or a
Complaint}\label{making-a-suggestion-or-a-complaint}

Students with a suggestion or complaint should first visit with the
instructor (and the course supervisor), and then with the departmental
DEO. Complaints must be made within six months of the incident
(\href{https://clas.uiowa.edu/students/handbook}{CLAS Academic Policies
Handbook}).

\subsection{Understanding Sexual
Harassment}\label{understanding-sexual-harassment}

Sexual harassment subverts the mission of the University and threatens
the well-being of students, faculty, and staff. All members of the UI
community have a responsibility to uphold this mission and to contribute
to a safe environment that enhances learning. Incidents of sexual
harassment should be reported immediately. See the UI
\href{https://osmrc.uiowa.edu/}{Office of the Sexual Misconduct Response
Coordinator} for assistance, definitions, and the full University
policy.

\subsection{Reacting Safely to Severe
Weather}\label{reacting-safely-to-severe-weather}

In severe weather, class members should seek appropriate shelter
immediately, leaving the classroom if necessary. The class will continue
if possible when the event is over. For more information on Hawk Alert
and the siren warning system, visit the
\href{https://police.uiowa.edu/emergency-communications}{Department of
Public Safety website}.

\clearpage

\section*{Reference}\label{reference}
\addcontentsline{toc}{section}{Reference}

\hypertarget{refs}{}
\hypertarget{ref-Babbie2012}{}
Babbie, Earl R. 2012. \emph{The Practice of Social Research}. 13th ed.
Australia: Wadsworth Cengage Learning.

\hypertarget{ref-VanEvera1997}{}
Van Evera, S. 1997. \emph{Guide to Methods for Students of Political
Science}. Cornell Paperbacks. Cornell University Press.



\clearpage
\end{document}

\makeatletter
\def\@maketitle{%
  \newpage
%  \null
%  \vskip 2em%
%  \begin{center}%
  \let \footnote \thanks
    {\fontsize{18}{20}\selectfont\raggedright  \setlength{\parindent}{0pt} \@title \par}%
}
%\fi
\makeatother
